
\subsection{\slurm{}}\label{subsec:slurm}

Since the data corpus is too big to be processed locally on a \localMaschineStats{}, the Chair \ac{ies} has offered to provide computational means to solve this problem.
The scripts can be processed by multiple nodes which are managed by \slurm{}.

\slurm{} is an open-source management tool for Linux clusters \cite{slurm-online}.
It allocates resources, i.e.\ compute nodes, and provides the means to start, execute and monitor jobs \cite{slurm-online, slurm2003}.

The so-called \slurm{} daemons control nodes, partitions, jobs and job steps \cite{slurm-online}.
A partition is a group of nodes and a job is the allocation of resources, i.e.\ compute nodes, to a user for a limited period of time.
A basic visualization of the architecture is given in \autoref{fig:slurm-architecture}.

\begin{figure}[!htb] % htp = hier (h), top (t), oder auf einer eigenen Seite (p).
    \centering
    \includesvg[width=0.7\textwidth]{images/slurm/slurm_architecture}
    \caption[\slurm{} architecture]{\slurm{} architecture. The management node has a \texttt{slurmctld} daemon, while every compute node has a \texttt{slurmd} daemon.
    The nodes communicate.
    The user can use certain commands, for instance \texttt{srun} and \texttt{squeue}, anywhere on the cluster.
    }
    \label{fig:slurm-architecture}
\end{figure}

% sbatch scripts
A job is started by a \texttt{sbatch} script.
This script defines the \texttt{partition}, the \texttt{job-name}, the number of \texttt{nodes}, the \texttt{cpus-per-task}, the memory \texttt{mem} allocated, 
the \texttt{time} limit and the path to store \texttt{error} and \texttt{output} logs.
It is possible to work on multiple \acp{cpu} simultaneously to divide the workload of a task.
In this work, multiple \texttt{sbatch} scripts are used to carry out a variety of tasks.
A summary of the tasks and scripts is given in \autoref{tbl:sbatch-scripts}.


\begin{table}[]
    \caption{A selection of \texttt{sbatch} scripts used in this work.}
    \begin{tabular}{|
    >{\columncolor[HTML]{EFEFEF}}p{0.28\textwidth} |p{0.68\textwidth}|}
    \hline
    \cellcolor[HTML]{C0C0C0}\textbf{Name of sbatch script} & \cellcolor[HTML]{C0C0C0}\textbf{Description}                                                                                                                                                                                                                      \\ \hline
    ae\_config.sh                                          & Comparison of different \ac{ae} architectures in terms of the metrics cosine similarity and \ac{rsme}.                                                                                                                                                                     \\ \hline
    allocate\_res.sh                                       & Allocates resources to enable a \ac{ssh} tunnel connection from a local \ac{vscode} instance to the server of the \ac{ies}. 
                                                                When enabled, the database content can be displayed with the \databaseName{} plugin.\\ \hline
    create\_database.sh                                    & Initializes the database by specifying fields.                                                                                                                                                                                                                    \\ \hline
    create\_documents.sh                                   & Inserts the document's metadata information, i.e.\ path and text.                                                                                                                                                                                                   \\ \hline
    elasticContainer.sh                                    & Starts the \databaseName{} container using the headless \texttt{podman-compose up} command.                                                                                                                                     \\ \hline
    init\_database.sh                                      & Initializes database, subsequentially inserts documents metadata, embeddings and clusters.                                                                                                                                                                        \\ \hline
    insert\_clusters.sh                                    & Inserts \ac{pca} weights, \ac{optics} and argmax clusters.                                                                                                                                                                      \\ \hline
    insert\_embeddings.sh                                  & Subsequentially inserts embeddings of documents.                                                                                                                                                                                                                  \\ \hline
    own\_w2v\_model.sh                                     & Creates and saves custom \ac{w2v} model.                                                                                                                                                                                                          \\ \hline
    run\_pdf2png.sh                                        & Converts and saves the \ac{png} version of the first page of the \acp{pdf}.                                                                                                                                                             \\ \hline
    \end{tabular}
    \label{tbl:sbatch-scripts}
\end{table}
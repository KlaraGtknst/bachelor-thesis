
\section{\slurm{}}\label{subsec:slurm}

Since the data corpus is too big to be processed locally on a \localMaschineStats{}, the \thesisdepartment{} has offered to provide computally means to solve this problem.
The scripts can be processed by multiple nodes which are managed by \slurm{}.

\slurm{} is an open-source management tool for Linux clusters \cite{slurm-online}.
It allocates resources, i.e. compute nodes, and provides the means to start, execute and monitor jobs \cite{slurm-online, slurm2003}.

The so-called \slurm{} daemons control nodes, partitions, jobs and job steps \cite{slurm-online}.
According to \citeauthor{slurm-online}, a partition is a group of nodes and a job is the allocation of resources, i.e. compute nodes, to a user for a limited period of time.
A basic visualization of the architecture is given in \autoref{fig:slurm-architecture}.

\begin{figure}[htp] % htp = hier (h), top (t), oder auf einer eigenen Seite (p).
    \centering
    \includesvg[width=0.7\textwidth]{images/slurm_architecture}
    \caption{\slurm{} architecture. The management node has a \texttt{slurmctld} daemon, while every compute node has a \texttt{slurmd} daemon.
    The nodes communicate.
    The user can use certain commands, for instance \texttt{srun} and \texttt{squeue}, anywhere on the cluster.
    }
    \label{fig:slurm-architecture}
\end{figure}
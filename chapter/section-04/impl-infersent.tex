\subsubsection*{\infersent{}}\label{subsubsec:impl-infersent}

% parameters
The \infersent{} model is provided by PyTorch \cite{HfsentTrans2019}.
The parameters used to initialize the model are presented in \lst{lst:infersent-params}.
The parameter \texttt{version} indicates whether the model is trained with \acs{glove} or fastText for the value 1 or 2 respectively.
Since the model is precomputed, it is not possible to change certain parameters, 
such as the word embedding dimension \texttt{word\_emb\_dim} or the dimension of the output vectors \texttt{enc\_lstm\_dim}.

\begin{listing}[htp]
    \begin{minted}{python3}
        {'bsize': 64, 'word_emb_dim': 300, 'enc_lstm_dim': 2048,  
        'pool_type': 'max', 'dpout_model': 0.0, 'version': 1}
    \end{minted}
    \caption{Parameters of the \infersent{} model.
    }
    \label{lst:infersent-params}
\end{listing}

% state dict
The steps necessary to create a working instance of the \infersent{} model are presented in \lst{lst:infersent-init}.
After the \infersent{} model is initialized in line \ref{line:init}, the \texttt{state\_dict} of the model is loaded in line \ref{line:state_dict}.
This dictionary consists of learnable parameters, i. e. weights and bias, of the model.
The path to the word embeddings is set in line \ref{line:w2v_path}.
Finally, in line \ref{line:vocab}, the vocabulary of the model is built or more precisely, only those embeddings needed are kept while the rest is discarded.

\begin{listing}[htp]
    \begin{minted}[escapeinside=||]{python3}
        infersent = InferSent(params_model)|\phantomsection\label{line:init}|
        infersent.load_state_dict(torch.load(model_path))|\phantomsection\label{line:state_dict}|
        infersent.set_w2v_path(w2v_path)|\phantomsection\label{line:w2v_path}|
        infersent.build_vocab(docs, tokenize=True)|\phantomsection\label{line:vocab}|
    \end{minted}
    \caption{Initializing the \infersent{} model.
    }
    \label{lst:infersent-init}
\end{listing}

% GloVe
Initially, in this work, the \infersent{} model was based on \acs{glove} word embeddings.
\acs{glove} is a global log-bilinear regression model for unsupervised learning of vector representations of words \cite{glove2014}. 
According to \citeauthor{glove2014}, the model captures global corpus statistics directly.
%The complexity of the model is bound by $O(V^2)$, $V$ being the vocabulary size.
%However, \citeauthor{glove2014} claim it is better than the worst case stated above and rather in $O(C)$, $C$ being the corpus size.
The \acs{glove} model is trained on ratios of co-occurrence probabilities of words from the corpus, 
which correlate with the relationship between the words. % \cite{glove2014}.
\citeauthor{glove2014} introduce rules for a weighting function to ensure neither rare nor frequent co-occurrences are overweighted.
It is possible to download embeddings computed by \acs{glove}, instead of using the algorithm to generate them.
The precomputed word embeddings are stored in a 5.65 \ac{gb} text file.
The file contains 840 B tokens and a vocabulary of 2.2 M cased 300-dimensional vector representations of words \cite{download-glove}.
According to \citeauthor{UniversalSentEnc2018}, \acs{glove} introduces bias in terms of ageism, racism and sexism into the model.

% own W2V
In this work, a custom set of vector representations of words is used.
The custom word embeddings are computed by a \ac{w2v} model trained on 2048 randomly selected documents from the Bahamas dataset.
The only parameter which differs from the default settings is the \texttt{vector\_size} which is set to 300.
After the \ac{w2v} model is trained, the word embeddings are saved in a file.
The file is post-processed to be compatible with the \infersent{} model.
To be more precise, only lines that consist of at least two whitespace-separated char sequences are kept.
Usually, word embeddings stored in a text file are structured in a way that 
the first char sequence is the word and the following numbers are the vector representation of the word.

% Autoencoder
In this work, an \ac{ae} is used to reduce the dimensionality of the \infersent{} embedding.
The implementation of the \ac{ae} is outlined below. %in \autoref{subsubsec:impl-autoencoder}.

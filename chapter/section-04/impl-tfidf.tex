\section{\ac{tfidf}}\label{sec:impl-tfidf}

% overall architecture
The \ac{tfidf} model has to be initialized and trained on the data corpus to build the data-specific vocabulary.
An exemplary implementation is given in \lst{lst:impl-tfidf}.
The \texttt{TfidfVectorizer} is provided by the \texttt{scikit-learn} package.
When initializing the model, the parameters define not only the input type but also the way the data is preprocessed.
The \texttt{input} parameter defines the input type, i.e. \texttt{content} means that the input is a list of strings or bytes, 
whereas \texttt{file} assumes the input has a \texttt{read} method and \texttt{filename} denotes a list of filenames as input \cite{tfidf-vec-scikit-learn}.

The \texttt{preprocessor} parameter defines the preprocessing (string transformation) stage.
It is possible to override the default with a custom preprocessing function.
The parameters \texttt{min\_df} and \texttt{max\_df} define the minimum and maximum document frequency of a word in the corpus.
The default values are $1$, i.e. a term has to appear at least once, and $1.0$, i.e. a term appears at most in all documents, respectively \cite{tfidf-vec-scikit-learn}.

By default, the \texttt{scikit-learn} implementation uses the \texttt{norm='l2'} parameter, i.e. the Euclidean norm \cite{tfidf-scikit-learn}.
The implementation of \ac{tfidf} in \texttt{scikit-learn} is different from the original \ac{tfidf} definition.
The difference is the calculation of the \ac{idf} part, which is given in \autoref{eq:tfidf-scikit-learn} from \cite{tfidf-scikit-learn}.
The one is added to $M_{ij}$ due to the parameter \texttt{smooth\_id=True} by default to prevent zero divisions \cite{tfidf-scikit-learn}
and to avoid logarithmic divergences due to a zero argument \cite{glove2014}.
After calculating the \ac{tfidf} values, they are normalized by the Euclidean norm 
$v_{norm} = \frac{v}{\left\| v \right\|_{2}} = \frac{v}{\sqrt{v_1^{2} + v_2^{2} + ... + v_M^{2}}}$.

\begin{equation}
    \text{idf}(w_{ij}) = \log \frac{1 + M}{1 + M_{ij}} + 1    
    \label{eq:tfidf-scikit-learn}
\end{equation}

\begin{listing}[htp]
    \begin{minted}{python3}
        tfidf = TfidfVectorizer(input='content', 
                    preprocessor=TfidfTextPreprocessor().transform, min_df=3, 
                    max_df=int(len(docs)*0.07))
        tfidf.fit(documents)
    \end{minted}
    \caption{Initialization of the \ac{tfidf} model.
    Firstly, an instance of the \texttt{TfidfVectorizer} class is created.
    Secondly, the \texttt{fit} method is called to fit the model on the documents.
    }
    \label{lst:impl-tfidf}
\end{listing}

% pipeline
In this work, the text of the \acp{pdf} is first extracted, then preprocessed using a custom preprocessor and afterwards embedded using the \texttt{TfidfVectorizer},
which returns the \ac{tfidf} weights as embedding.
Before storing the \ac{tfidf} weights in the database, they are enhanced with an all-zero flag.
The all-zero flag ensures that no all-zero vectors are stored in the database by extending those that have a zero magnitude with a "1" entry and "0" otherwise.
All-zero \ac{tfidf} weights indicate that a document does not have any terms with the vocabulary in common.
Since the vocabulary is kept relatively small with respect to the number of different words in the data corpus to reduce the dimensionality of the embeddings, 
it is not unlikely that a document does not contain any of the vocabulary terms.
The all-zero flag is necessary because the cosine similarity used to query for similar documents in the database cannot handle vectors of zero magnitude.
The pipeline in \autoref{fig:tfidf_embedding} visualizes the steps.

\begin{figure}[h] % htp = hier (h), top (t), oder auf einer eigenen Seite (p).
    \centering
    \includesvg[width=1.0\textwidth]{images/TFIDF_embedding}
    \caption{\ac{tfidf} pipeline.
    Firstly, the text extracted from the documents is preprocessed using a custom preprocessor.
    Then, the \ac{tfidf} values are obtained from the \texttt{TfidfVectorizer}.
    Lastly, the all-zero flag is added to the \ac{tfidf} weights and they are stored in the database \textcolor{red}{if dim $<$ 2048, ae otherwise}.
    }
    \label{fig:tfidf_embedding}
\end{figure}

% custom preprocessing
In this work, a custom preprocessing function is used.
The preprocessing steps are visualized in \autoref{fig:preprocessing} on an exemplary text.
Firstly, the accents are stripped from the text.
Then, all new line symbols are replaced with a whitespace.
Afterwards, the text is converted to lowercase.
Then the numbers are discretized, i.e. all numbers between 0 and 99999 are replaced with the string \texttt{SMALLNUMBER}, 
numbers bigger than 99999 are replaced with the string \texttt{BIGNUMBER} and floats are replaced with the string \texttt{FLOAT}.
The next step is to remove all punctuation symbols.
After that, the symbols for numbers are enclosed with pointed brackets, e.g. \texttt{<SMALLNUMBER>}.
Then, the text is tokenized, i.e. splitting at whitespaces, and stop words are omitted.
\textcolor{red}{omitting double space missing}
The \texttt{nltk} package provides the stop words used to omit uniformative words.
The stop word corupus consists of common English stop words.
Afterwards, the tokens are lemmatized.
The lemmatizer used is \texttt{WordNetLemmatizer} from the \texttt{nltk} package.
In the end, the tokens are joined to a string again.


\begin{figure}[htp] % htp = hier (h), top (t), oder auf einer eigenen Seite (p).
    \centering
    \includesvg[width=1.1\textwidth]{images/TFIDF_preprocessing}
    \caption{\ac{tfidf} Preprocessing visualized using a example text. \textcolor{red}{TODO: nicht aktuell bzgl. small number - etc., remove double spaces fehlt auch}}
    \label{fig:preprocessing}
\end{figure}
\subsubsection*{\acl{ae}}\label{subsubsec:impl-autoencoder}

In this work, an \ac{ae} is used to reduce the dimensionality of the \infersent{} and the \ac{tfidf} embeddings.
Since the \infersent{} model is pretrained, it is not possible to change the dimensionality of the embedding without a considerably big effort,
i.e.\ retraining the model on a sufficiently large data corpus and reconfiguring the model's parameters.
Therefore, it is not feasible to change the dimensionality of the \infersent{} embedding, but rather add a supplementary layer after the model 
to produce the final embedding.
Similarly, the \ac{tfidf} embedding dimension correlates with the vocabulary size and thus, the size of the data corpus.
Further reducing the vocabulary size would decrease the \ac{tfidf} model's quality.
Hence, the idea is to use the encoder of an \ac{ae} to reduce the dimensionality of the \infersent{} and the \ac{tfidf} embedding.

\begin{figure}[!htp] % htp = hier (h), top (t), oder auf einer eigenen Seite (p).
    \centering
    \includesvg[width=1.0\textwidth]{images/embeddings/autoencoder/autoencoder-impl.svg}
    \caption[Architecture of the \acs*{ae}]{Architecture of the \acs*{ae}.}
    \label{fig:impl-ae}
\end{figure}

The implementation was provided by the blog post from \cite{impl-src-ae}.
It uses the library keras\footnote{https://keras.io/ (last accessed: 19/11/2023)}.
The architecture is adapted to fulfil the needs of the specific context.
It is presented in \autoref{fig:impl-ae}.
\subsubsection*{\ac{t2v}}\label{subsubsec:impl-top2vec}

\citeauthor{Top2Vec2020}'s \ac{t2v} model is provided in the Python library \ac{t2v} \cite{Top2Vec2020}.
The non-changeable default values and settings of the model are listed below.
The word and document embeddings are generated by the \ac{d2v} version \ac{pvdbow}.
It has a window size of 15, uses hierarchical softmax, a minimum count of 50, a vector size of 300 and a sub-sampling threshold of $10^5$.
% dimensionality reduction
\ac{umap}'s hyperparameters are set to 15 nearest neighbours, 
cosine similarity as the distance metric and 5 as the embedding dimension in \citeauthor{Top2Vec2020}'s work.

In this work, a class is implemented, which uses the \ac{t2v} library.
When initiating an instance of this class, the \ac{t2v} model is trained on the given document corpus as displayed in \lst{lst:init-top2vec}.
The class provides methods to query for the number of topics as well as the 
most similar topics and documents to an input keyword.
The most similar topics can be visualized using \wordcloud{}s.
The core functionalities are implemented by the \ac{t2v} library, 
but the class is used to modify the return values to be compatible with the \ac{ui}.


\begin{listing}[htp]
    \begin{minted}{python3}
        Top2Vec(documents=self.documents, speed='fast-learn', workers=8)
    \end{minted}
    \caption[Initialization of the \ac{t2v} model]
    {Initialization of the \ac{t2v} model.
    }
    \label{lst:init-top2vec}
\end{listing}
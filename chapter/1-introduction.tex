\chapter{Einleitung}\label{ch:introduction}

Das ist die Einleitung.
"Dies ist ein Zitat"~\cite{dragon-book}.
Das ist \ac{zB} eine Abkürzung: \ac{http}.
Beim zweiten Mal steht nur noch \ac{http}.
Das ist eine Fußnote\footnote{Ich putz hier nur.}.

Abbildung~\ref{fig:uni-kassel-logo} zeigt das Logo der Uni Kassel.

\begin{figure}[htp] % htp = hier (h), top (t), oder auf einer eigenen Seite (p).
    \centering
    \includegraphics[width=0.5\textwidth]{images/Logo_UniKassel.png} % width immer angeben!
    \caption{Das Logo der Uni Kassel}
    \label{fig:uni-kassel-logo}
\end{figure}

Listing~\ref{lst:simple-class} implementiert eine Klasse in \code{java}.

\begin{listing}[htp]
    \begin{minted}{java}
        class Foo {
            String bar;
        }
    \end{minted}
    \caption{Eine einfache Klasse}
    \label{lst:simple-class}
\end{listing}

Tabelle~\ref{tbl:evaluation-data} enthält die Daten für die Auswertung.

\begin{table}[htp]
    \centering
    \caption{Einfache Daten}
    \begin{tabular}{|l|l|l|l|}
    \hline
        Nr. & Punkte & Aufgaben & Bewertet \\
        \hline
        1  & 30 & 40 & 26 \\
        2  & 44 & 75 & 43 \\
        3  & 22 & 23 & 14 \\
        4  & 47 & 46 & 32 \\
        5  & 45 & 63 & 42 \\
        6  & 58 & 71 & 54 \\
        7  & 54 & 80 & 54 \\
        8  & 51 & 60 & 44 \\
        9  & 35 & 48 & 35 \\
        10 & 25 & 38 & 25 \\
        11 & 37 & 48 & 37 \\
        \hline
        Gesamt & 448 & 592 & 406 \\
        \hline
    \end{tabular}
    \label{tbl:evaluation-data}
\end{table}
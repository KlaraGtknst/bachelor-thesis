\section{Topic analysis}\label{sec:topic-modeling}

Since more and more textual data emerges, methods to analyze and extract information from texts become more important.
One of these methods is topic analysis.
%It is used to discover groups of words with similar meanings in a text corpus \cite{topic_modeling2015}.
A topic can be defined as a cluster of words that occur frequently.
Subsequently, it is described by a probability distribution over a vocabulary.
A document can be represented by one or more topics. %and thus, can be created by a distribution over topics and a distribution over words given the topics.
%According to \citeauthor{topic_modeling2015}, topic models ignore the order of words in a document by relying on the bag-of-words assumption.
Common topic analysis algorithms include \ac{lda} \cite{topic_modeling2015}.

\subsection{\acl*{lda}}\label{subsec:latent-dirichlet-allocation}

\ac{lda} is a generative model, which recreates the original document word distributions with minimal error given the topics \cite{topic_modeling2015, Top2Vec2020}.
A discrete multinomial probability distribution over a vocabulary consisting of $W$ words is called a topic.
A document is a mixture of $K$ latent topics.
$K$ is defined a priori by the user.
Hence, each document from the document corpus $D$ is represented by specific topic probabilities.

The probability distribution learned by \ac{lda} only considers the statistical relationship of word occurrences in documents \cite{Topic2Vec2015}.
The content of a document can be described by the top $N$ words with the highest conditional probability given a topic \cite{Topic2Vec2015}.
Due to the fact that \ac{lda} considers high probability words to be informative, words 
that occur frequently are deemed important even though they might be uninformative in reality \cite{Top2Vec2020, Topic2Vec2015}.


\subsection{Top2Vec}\label{subsec:top2vec}

The approach \topTwovec{} was proprosed by \citeauthor{Top2Vec2020} \cite{Top2Vec2020}.
It addresses several problems of state-of-the-art topic modelling approaches, such as \ac{lda}.
Opposed to \ac{lda}, \topTwovec{} does not require the user to specify the number of topics $k$, 
i.e. it does not discretize the topic space into $k$ topics, 
and it does not require stop word removal or lemmatization.
Moreover, it considers the semantic meaning of words unlike \ac{lda}.
In contrast to \ac{lda}, \topTwovec{} only associates one topic with a document.

\begin{figure}%
    \centering
    \subfloat[\centering Skip-gram architecture cf. \cite{Topic2Vec2015}.]
    {{\includesvg[width=5.5cm]{images/topic_modelling/Skip-gram_Top2Vec.svg}}}%
    \qquad
    \subfloat[\centering \ac{cbow} architecture cf. \cite{Topic2Vec2015}.]
    {{\includesvg[width=5.5cm]{images/topic_modelling/CBOW_Top2Vec.svg} }}%
    \caption[Two learning architectures of \topTwovec{}]{Both learning architectures of \topTwovec{}.
    $w(t-2), w(t-1), w(t+1), w(t+2)$ are the context words of the centre word $w(t)$ of topic $z(t)$.
    }%
    \label{fig:top2vec_architectures}%
\end{figure}

% word2vec and doc2vec
\topTwovec{} is based on \ac{w2v} and \ac{d2v}.
The documents are embedded using \ac{pvdbow}.
There are two learning architectures adapted from \ac{w2v} to train the model, namely \ac{cbow} and Skip-gram \cite{Topic2Vec2015}.
They are depicted in \autoref{fig:top2vec_architectures}.
The Skip-Gram learning task is to predict the document a word came from \cite{Top2Vec2020, Topic2Vec2015}.
Similar to \ac{d2v}, \topTwovec{} not only embeds words and documents in the same feature space but also topics \cite{Top2Vec2020, Topic2Vec2015}.
The similarity between embeddings can be measured using the cosine similarity function \cite{Topic2Vec2015}.
\citeauthor{Top2Vec2020} regards each point in the \ac{vsm} as a topic, described by its nearest words.

% topics & clustering
\citeauthor{Top2Vec2020} states that topics are continuous and can be described by different sets of words \cite{Top2Vec2020}.
Hence, topic modelling is defined as the task of finding sets of informative words that describe a document.
Documents in dense areas of the topic space are considered to be about the same topic.
The density-based clustering algorithm \ac{hdbscan} is used to find these dense areas.
The topic vector is denoted as the centroid or average of the document vectors belonging to the topic.
\citeauthor{Top2Vec2020} has compared several topic vector definitions and found the arithmetic mean to be the best option \cite{Top2Vec2020}.
The number of topics is derived from the number of dense areas.
It is possible to merge topics to hierarchically reduce the number of topics found to any number of topics smaller than the number initially found.

% dimensionality reduction
In order to find topics, clusters of documents, i.e. dense areas, need to be identified.
The clustering algorithm \ac{hdbscan} is used to find these dense areas.
Since \ac{hdbscan} has difficulties finding dense clusters in high-dimensional data, 
the dimensionality reduction method \ac{umap} is applied \cite{Top2Vec2020}.
The steps of the topic modelling procedure \topTwovec{} are depicted in \autoref{fig:top2vec}.

% complexity
According to \citeauthor{Topic2Vec2015}, the complexity of \topTwovec{} is linear with the size of the dataset \cite{Topic2Vec2015}.

\begin{figure}[htp] % htp = hier (h), top (t), oder auf einer eigenen Seite (p).
    \centering
    \includesvg[width=0.7\textwidth]{images/topic_modelling/Top2Vec.svg}
    \caption{Procedure of topic modelling using \topTwovec{}.}
    \label{fig:top2vec}
\end{figure}



\subsection{\wordcloud{}s}\label{subsec:word-clouds}

A \wordcloud{} is a technique to visualize the most predominant words in a text \cite{topic_modeling2019}.
The size of a word correlates to its frequency or importance in the text.
However, a word does not have to be meaningful to appear large.
A \wordcloud{} does not provide information about the meaning or context of words and thus, 
one has to be careful when interpreting the results.
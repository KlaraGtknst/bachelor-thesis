
\section{Elasticsearch}\label{subsec:impl-db}

% content
In this work, the database is filled once with data from a large unstructured corpus of \ac{pdf} files.
After the initialization of the database, it is used for queries. 
Therefore, the workflow is completely offline.

The index \textit{Bahamas} stores different embeddings of the text layer information and metadata of the documents.
As depicted in \autoref{fig:pdf2db}, not only textual information is stored in the database, but also the images of the first page of the \acp{pdf}.
The structure of the index is presented in \autoref{tbl:Elasticsearch-fields}.

% TODO: add field type or dimension?
\begin{table}[]
    \caption{Fields in \databaseName{} database in index \textit{Bahamas}.}
    \begin{tabular}{|
    >{\columncolor[HTML]{EFEFEF}}l |p{0.63\textwidth}|}
    \hline
    \cellcolor[HTML]{C0C0C0}\textbf{field name} & \cellcolor[HTML]{C0C0C0}\textbf{field description}                                     \\ \hline
    \_id                                        & Unique identifier of document \texttt{i}. The identifier is generated by the sha256 hash algorithm from hashlib.\\ \hline
    doc2vec                                     & 55 dimensional doc2vec embedding of \texttt{i}.                                                          \\ \hline
    sim\_docs\_tfidf                            & sim\_docs\_tfidf embedding + all-zero flag of \texttt{i}. The all-zero flag is one if the \ac{tfidf} embedding consists of only zeros, zero else.\\ \hline
    google\_univ\_sent\_encoding                & 512 dimensional google\_univ\_sent\_encoding embedding of \texttt{i}.                                     \\ \hline
    huggingface\_sent\_transformer              & 384 dimensional huggingface\_sent\_transformer embedding of \texttt{i}.                                  \\ \hline
    inferSent\_AE                               & inferSent\_AE embedding of \texttt{i}. Since the pretrained infersent model embedding's dimension is 4096, the encoder of a trained \ac{ae} is added to reduce the dimension to 2048.                                                    \\ \hline
    pca\_image                                  & Two dimensional \ac{pca} version of first page image of \texttt{i}.                      \\ \hline
    pca\_kmeans\_cluster                        & Cluster of \texttt{i} identified by KMeans on \ac{pca} version of image.                 \\ \hline
    text                                        & Text of \texttt{i}.                                                                       \\ \hline
    path                                        & Path on local maschine to \texttt{i}.                                                     \\ \hline
    image                                       & Base64 encoded image of first page of \texttt{i}.                                                        \\ \hline
    \end{tabular}
    \label{tbl:Elasticsearch-fields}
\end{table}

\begin{figure}[htp] % htp = hier (h), top (t), oder auf einer eigenen Seite (p).
    \centering
    \includesvg[width=0.7\textwidth]{images/PDFs_to_database}
    \caption{\acp{pdf} to Database. 
    First, the data is preprocessed:
    The first page of a \ac{pdf} file is converted to an image and the complete text is extracted. 
    The images are stored in the database as well as the text and different embeddings of the text.
    }
    \label{fig:pdf2db}
\end{figure}

The default analyzer is used for the full-text search, since for instance configuring a maximum token length did not seem necessary or likely to improve the results.

\begin{listing}[htp]
    \begin{minted}{python3}
        results = elastic_search_client.search(
            index='bahamas', 
            size=count,
            from_=(page*count),
            query= {'match' : {
                        'text': {   'query':text,
                                    'fuzziness': 'AUTO',}
                    }, 
                }, source_includes=SRC_INCLUDES)
    \end{minted}
    \caption{Exemplartary query to an \databaseName database index.
    The number of results to return \texttt{size} and the start index of the results \texttt{from\_} is defined.
    To enable fuzzy search a value for \texttt{fuzziness} has to be defined. 
    }
    \label{lst:fuzzy_query}
\end{listing}

Moreover, the fuzzy matching option is set to \texttt{AUTO}, which means in terms of keyword or text fields that the allowed Levenshtein Edit Distance, 
i.e. number of characters changed to create an exact match between two terms, to be considered a match, is correlated to the length of the term \cite{Elasticsearch-fuzziness}.
By default, terms of length up to two characters must match exactly, terms of length three to five characters must have an edit distance of one and 
terms of length six or more characters must have an edit distance of two \cite{Elasticsearch-fuzziness}.
An exemplartary query, which uses fuzzy search is given in \lst{lst:fuzzy_query}.

According to \citeauthor{Elasticsearch-kNN-HNSW}, one of \ac{knn} search's use cases is semantic document retrieval, which makes it a good fit for this task.
In this work, the approximate nearest neighbors search is used, since it is faster and the results are good enough for the use case of this work.
The similarity measure used in this work is the cosine similarity, which calculates the \texttt{\_score} of a document according to \autoref{eq:cosine-similarity-db} from \cite{Elasticsearch-kNN-similarity}, 
where \texttt{query} is the query vector and \texttt{vector} is the vector representation of the document in the database.
Since cosine is not defined on vectors with zero magnitude, embeddings that can return all zero vector representations, such as sim\_docs\_tfidf, 
are enhanced with an all-zero flag before inserting them into the database.

\begin{equation}
    \frac{1 + \text{cosine}(\text{query}, \text{vector})}{2}
    \label{eq:cosine-similarity-db}
\end{equation}

In this work, the only tool from the elastic stack used is \databaseName{}.
Without Kibana, the used models are saved on disk as \ac{pkl} files.
Consequently, instead of using the \ac{knn} query structure for semantic search on embeddings provided by \databaseName{}, the normal \ac{knn} search on a field that contains an embedding is used.


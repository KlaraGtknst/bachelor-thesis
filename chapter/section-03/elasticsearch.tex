
\subsection{Elasticsearch database}\label{sec:db}

% introduction, users
\databaseName{} is a widely used non-relational database, which was designed to store and perform full-text search on a large corpus of unstructured data \cite{Elasticsearch2017}.
This open-source distributed document-driven database system is built in Java and is based on the Apache Lucene (Java) library for high-speed full-text search \cite{Elasticsearch2017, Elasticsearch2019}.
According to \citeauthor{Elasticsearch2019}, \databaseName{} provides Wikipedia's full-text search 
and suggestions as well as Github's code search and Stack Overflow's geolocation queries and related questions.
It enables near real-time search by short refreshing periods which make performed operations on the data quickly available for search.
%Needless to say, \databaseName{} is qualified to handle Big Data.

% structure
\databaseName{} is a document store, which stores schemaless key-value pairs called documents \cite{flask2018}.
The documents are stored in logical units, so-called indices.
% index
As stated by \citeauthor{Elasticsearch2019} and \citeauthor{Elasticsearch2017}, the indices are structured similarly to Apache Lucene's inverted index format.
An index can be spread into multiple nodes.
A node is a single running instance of \databaseName{} \cite{Elasticsearch2019}.
An index is divided into one or more shards, which can be stored on different servers and enable parallelization.
% Replicas
Replicas are copies of shards, which create redundancy and thus, ensure availability. %\cite{Elasticsearch2019}.

% document
The documents are saved in a \ac{json} format \cite{Elasticsearch2017}.
A document's fields and field types are defined by the user when initializing the database index.
By default, every field of a document is indexed and searchable \cite{Elasticsearch2019}.

% query (endpoints)
% get: search id
By specifying the unique \texttt{\_id} of a document and the database \texttt{index}, 
it is possible to retrieve a specific document from the database using a \texttt{GET} endpoint of the \ac{http} \ac{api}.
%The query is real-time by default.
The parameters \texttt{\_source\_excludes} or \texttt{\_source\_includes} can be used to define the structure of the response \cite{Elasticsearch-get}.

% full-text search
The keyword used when performing a full-text search is \texttt{match}.
To query for a specific value, one has to specify the field of interest and the query value.

\databaseName{} preprocesses the query value before starting the search \cite{Elasticsearch-text-analyser}.
The default preprocessing steps of the so-called default analyzer include tokenization and lowercasing \cite{Elasticsearch-text-analyser}. 
Omitting stop words is disabled by default, but it is possible to provide custom stop words or use the English stop word list \cite{Elasticsearch-text-analyser}.
It is possible to create custom tokenizers, which split the query value into tokens of a certain maximum length.

Another useful feature of \databaseName{} is the multi-term synonym expansion.
When the user queries a specific phrase, \databaseName{} expands the query to include synonyms of the query terms \cite{Elasticsearch-synonyms}.
The maximum number of expansion terms is set to 50 by default but can be configured by the user \cite{Elasticsearch-match}.
By default, the multi-term synonym expansion option is enabled.

\databaseName{} also provides the option to perform fuzzy matching instead of exact search.
By enabling the fuzzy matching option, a \databaseName{} query consisting of, for instance, \textit{Bahama} returns documents that contain the word \textit{Bahamas}.
By default, this option is not enabled but can be enabled and configured individually by the user \cite{Elasticsearch-match}.


% knn-search
Another search option of \databaseName{} is the \ac{knn} search on real-valued vectors.
The return value of a \ac{knn} search is the \texttt{k} nearest neighbours in terms of a certain distance function of a query vector \cite{Elasticsearch-kNN-HNSW}.
In order to perform \ac{knn} search on a field it has to be of type \texttt{dense\_vector}, 
indexed and a \texttt{similarity} measure has to be defined when initializing the database \cite{Elasticsearch-knn}.
The query has the same dimension as the vectors stored in the database.
A \ac{knn} search either returns the exact brute-force nearest neighbours or 
an approximation of the nearest neighbours calculated by the \ac{hnsw} algorithm \cite{Elasticsearch-kNN-HNSW, Elasticsearch-knn}.
\ac{hnsw} is a graph-based algorithm \cite{Elasticsearch-kNN-HNSW}.

% The term \texttt{navigable} refers to the graphs used, which are graphs with (poly-)logarithmic scaling of links traversed 
% during greedy traversal concerning the network size \cite{Elasticsearch-kNN-HNSW}.
% The idea of a \texttt{hiercharical} algorithm is to create a multilayer graph, grouping links according to their link length, as displayed in \autoref{fig:hnsw-layer}. 
% The search starts on the uppermost layer, i.e.\ the layer containing the longest links, greedily traversing the layer until reaching the local minimum.
% It uses this local minimum as the starting point at the next lower layer and the process is repeated until the lowest layer is reached \cite{Elasticsearch-kNN-HNSW}.
% The layers of the graph are built incrementally, and a neighbour selection heuristic, as depicted in \autoref{fig:hnsw-heuristic}, not only creates links between close elements, 
% but also between isolated clusters to ensure global connectivity \cite{Elasticsearch-kNN-HNSW}.

% \begin{figure}[!htb] % htp = hier (h), top (t), oder auf einer eigenen Seite (p).
%     \centering
%     \includegraphics[width=0.4\textwidth]{images/Elasticsearch/HNSW-layer.png}
%     \caption[Structure of \ac{hnsw} layers]{Structure of \ac{hnsw} layers from \cite{Elasticsearch-kNN-HNSW}.
%     The search starts on the uppermost layer, i.e.\ the layer containing the longest links, greedily traversing the layer until reaching the local minimum.
%     The local minimum is used as the starting point at the next lower layer and the process is repeated until the lowest layer is reached.
%     }
%     \label{fig:hnsw-layer}
% \end{figure}

% \begin{figure}[!htb] % htp = hier (h), top (t), oder auf einer eigenen Seite (p).
%     \centering
%     \includegraphics[width=0.5\textwidth]{images/Elasticsearch/HNSW-neighbour-selection-heuristic.png}
%     \caption[Neighbour selection heuristic of \ac{hnsw}]{Neighbour selection heuristic of \ac{hnsw} from \cite{Elasticsearch-kNN-HNSW}.
%     The heuristic creates diverse links, i.e.\ links between close elements (e.g., green circle and elements in cluster 1) 
%     and between isolated clusters (e.g., green circle and $e_2$) to ensure global connectivity.
%     }
%     \label{fig:hnsw-heuristic}
% \end{figure}


Besides \databaseName{}, the elastic stack offers other tools, for instance, Kibana, which provides a user interface to manage different models.
After saving a model in Kibana, it is possible to create a text embedding ingest pipeline, which embeds new documents or reindexes existing documents \cite{Elasticsearch-knn-embedding}.
\databaseName{}'s \ac{knn} implementation not only allows literal matching on search terms 
but also semantic search incorporating Kibana's text embedding ingest pipeline on search terms \cite{Elasticsearch-knn}.
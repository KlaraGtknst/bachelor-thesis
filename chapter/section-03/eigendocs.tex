\section{\eigendocs{}}\label{subsec:eigendocs}

% this work
\begin{figure}[htp] % htp = hier (h), top (t), oder auf einer eigenen Seite (p).
    \centering
    \includesvg[width=1.0\textwidth]{images/eigendocs}
    \caption{From \acp{pdf} to \eigendocs{}.
    Firstly, the first page of a document is converted to an image.
    Then the image is preprocessed:
    It is placed on a white canvas, to ensure all images have the same dimensions.
    Moreover, it is converted to greyscale.
    Afterwards, the 2d image is reshaped to a 1d array.
    Lastly, the image is compressed using \eigenfaces{}.
    }
    \label{fig:eigendocs_procedure}
\end{figure}

In this work, the \eigenfaces{} approach from \autoref{subsec:eigenface} is used to compress the images of the first page of documents.
The idea is that documents not only hold textual information but also visual information, such as layout, company logo or signature.
By mapping those images on a subspace, they ought to be grouped by visual similarity.
The procedure of the eigenface adaption \textit{eigendocs} is displayed in \autoref{fig:eigendocs_procedure}.

% procedure
The documents are first read from a directory. 
Subsequently, their first page is converted to an image and saved.
When initially filling the database, these images are read from their directory.
Firstly, the maximum height and width of all images in the corpus is calculated.
These dimension are used to create a white canvas for each image which forms the background.
Every image is placed in the upper left corner.
Hence, scaling is not necessary and thus, the portion of white pixels on the right and bottom side encodes the dimension of the former image.
Therefore, the relative size of images in the corpus is incooperated in the resulting representation fo the input images.

\begin{listing}[htp]
    \begin{minted}{python3}
        C = np.ones((max_w,max_h))
        C[:doc.shape[0],:doc.shape[1]] = rgb2gray(doc)
        documents.append(C.ravel())
    \end{minted}
    \caption{Preprocessing of the input images from \thesissupervisor{}.
    The background is a white canvas.
    The images are converted to one-dimensional greyscale values.}
    \label{lst:preproc_images}
\end{listing}

Afterwards, they are converted to greyscale images using \autoref{lst:rgb2grey}.
Before returning the image, the two-dimensional image vectors are converted to one-dimensional ones as displayed in the last line of \autoref{lst:preproc_images}.
The decomposition is transformed using \ac{pca} as displayed in \autoref{lst:pca_svd}.

\begin{listing}[htp]
    \begin{minted}{python3}
        0.299*img[:,:,0] + 0.587*img[:,:,1] + 0.114*img[:,:,2]
    \end{minted}
    \caption{Conversion of RGB pixel values to greyscale by a script from \thesissupervisor{}.}
    \label{lst:rgb2grey}
\end{listing}

\begin{listing}[htp]
    \begin{minted}{python3}
        pca = decomposition.PCA(n_components=n_components, whiten=True, 
            svd_solver="randomized")
    \end{minted}
    \caption{Initialization of the \ac{pca} instace used to compress the image data.
    In order to work according to the \eigenfaces{} approach a svd\_solver has to be used.
    }
    \label{lst:pca_svd}
\end{listing}



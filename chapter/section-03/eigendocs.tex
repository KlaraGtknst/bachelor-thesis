\section{\eigendocs{}}\label{subsec:eigendocs}

% this work
\begin{figure}[htp] % htp = hier (h), top (t), oder auf einer eigenen Seite (p).
    \centering
    \includesvg[width=1.0\textwidth]{images/eigendocs}
    \caption{From \acp{pdf} to \eigendocs{}.
    Firstly, the first page of a document is converted to an image.
    Then the image is preprocessed:
    It is placed on a white canvas, to ensure all images have the same dimensions.
    Moreover, it is converted to greyscale.
    Afterwards, the 2d image is reshaped to a 1d array.
    Lastly, the image is compressed using \eigenfaces{}.
    }
    \label{fig:eigendocs_procedure}
\end{figure}

In this work, the \eigenfaces{} approach from \autoref{subsec:eigenface} is used to compress the images of the first page of documents.
The idea is that documents not only hold textual information but also visual information, such as layout, company logo or signature.
By mapping those images on a subspace, they ought to be grouped by visual similarity.
The procedure of the eigenface adaption \textit{eigendocs} is displayed in \autoref{fig:eigendocs_procedure}.

% procedure
\textcolor{red}{white canvas encodes information about the size of the document} 
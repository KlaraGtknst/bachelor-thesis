
\section{Similarity Measurement}\label{sec:similarity-measurement}

Since embeddings represent texts as numerical vectors, they not only facilitate human interpretability of relationships between texts using 
the text's respective point in a $N$-dimensional space, 
but they also enable the use of metrics, i.e. similarity measures, to quantify the similarity between texts \cite{IR2011, euclidean_l2_norm2015}.

There are several similarity measures, such as the dot product quantifying the number of shared tokens of two texts, 
the (soft) cosine similarity, which is the normalized dot product and calculates the angle between two vectors, 
and many more \cite{IR2011, euclidean_l2_norm2015, HfsentTrans2019}.
The following section outlines a few similarity measures.


\subsection{Euclidian distance}\label{subsec:euclidian-distance}

The \textit{euclidian distance} is a distance measure.
In order to measure the distance between two points in a $N$-dimensional space, 
the root of the sum of squared distances between the respective values of every dimension is calculated.
The distance function Euclidean (L2) norm is given in \autoref{eq:euclidian-distance} c.f. \cite{euclidean_l2_norm2015}.
The points $a, b$ correspond to objects $d_1, d_2$.

\begin{equation}
    d_E(a,b) = \sqrt{\sum_{i=1}^{N}(a_i - b_i)^2}
    \label{eq:euclidian-distance}
\end{equation}


\subsection{Cosine Similarity}\label{subsec:cosine-similarity}

In the traditional bag-of-words approach the texts are represented as vectors of \ac{tfidf} coefficients \cite{soft_cosine2017}.
Without further processing, the vector is of size $N$, $N$ being the number of different words of the texts \cite{soft_cosine2017}.
Hence, a vector represents its corresponding text in a $N$-dimensional space.
This space is called \ac{vsm} \cite{soft_cosine2014}.

The similarity between two texts is measured by the cosine of the angle between their respective vectors \cite{soft_cosine2014}.
The cosine similarity is defined in \autoref{eq:cosine-similarity} from \cite{soft_cosine2014}.
The dot-product is defined as $a \cdot b = \sum_{i=1}^{N}a_{i}b_{i}$.
It is normalized with $\left\| x \right\| = \sqrt{x \cdot x}$ to unit Euclidean length \cite{soft_cosine2014}.
The cosine similarity is a value between $0$ and $1$ for positive values \cite{soft_cosine2014}.
According to \citeauthor{soft_cosine2014}, the formula has a time and space complexity of $O(N)$ for a pair of $N$-dimensional vectors.

\begin{equation}
    cosine(a,b) = \frac{a \cdot b}{\left\| a \right\| \times \left\| b \right\|} = \frac{\sum_{i=1}^{N}a_{i}b_{i}}{\sqrt{\sum_{i=1}^{N}{a}^2_{i}}\sqrt{\sum_{i=1}^{N}{b}^2_{i}}}
    \label{eq:cosine-similarity}
\end{equation}

The formula \autoref{eq:cosine-similarity} assumes that the vectors, which span the \ac{vsm} are orthogonal and thus, 
completely independent \cite{soft_cosine2014}.
However, in practical applications, this often is not the case \cite{soft_cosine2014}.


\subsection{Soft Cosine Similarity}\label{subsec:soft-cosine-similarity}

This similarity measure not only evaluates whether two texts consist of the same words but 
also takes into account the semantic (word-level) similarity or lexical relation of different words of the texts \cite{soft_cosine2017}.
Hence, it improves the shortcomings of the traditional cosine similarity measure, 
which assumes the tokens of the vocabulary are completely independent of each other \cite{soft_cosine2014}.

According to \citeauthor{soft_cosine2014}, in order to model this additional information, more dimensions are added to the \ac{vsm}.
These dimensions can be obtained, for instance, by multiplying the mean of two features of one vector with the similarity between them \cite{soft_cosine2014}.
The similarity can be calculated by using Levenshtein distance, i.e. the number of operations necessary to convert one string into another, 
or using a dictionary of synonyms \cite{soft_cosine2014}.

Since this approach no longer assumes that different words are independent of each other, 
the basis vectors which span the \ac{vsm} are no longer orthogonal \cite{soft_cosine2014}.
The formula for the soft cosine similarity is defined in \autoref{eq:soft-cosine-similarity} from \cite{soft_cosine2014}.
The similarity $s_{ij}$ between the $i$-th and $j$-th basis vector is obtained using a similarity measure, such as synonymy \cite{soft_cosine2014}.

\begin{equation}
    soft\_cosine(a,b) = \frac{\sum_{i=1}^{N}\sum_{j=1}^{N}s_{ij}a_{i}b_{j}}{\sqrt{\sum_{i=1}^{N}\sum_{j=1}^{N}s_{ij}a_{i}a_{j}}\sqrt{\sum_{i=1}^{N}\sum_{j=1}^{N}s_{ij}b_{i}b_{j}}}
    \label{eq:soft-cosine-similarity}
\end{equation}

According to \citeauthor{soft_cosine2017}, the similarity between two texts is non-zero as soon as they share related words \cite{soft_cosine2017}.
If there is no similarity between different features, 
the soft cosine similarity from \autoref{eq:soft-cosine-similarity} is equal to the cosine similarity from \autoref{eq:cosine-similarity}.
The time and space complexity of the soft cosine similarity is $O(N^2)$ \cite{soft_cosine2014}.

In order to reduce the complexity, \citeauthor{soft_cosine2014} propose to use a sparse similarity matrix which only stores $s_{ij} > t$, 
$t$ being a threshold \cite{soft_cosine2014}.
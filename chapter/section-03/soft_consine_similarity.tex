
\subsection{Soft Cosine Similarity}\label{subsec:soft-cosine-similarity}

Soft cosine similarity assumes that features, i.e. index terms, correspond to words and thus, 
require embeddings to be produced by \ac{bow} models. 
This similarity measure not only evaluates whether two texts consist of the same words but 
also takes into account the semantic (word-level) similarity or lexical relation of different words of the texts \cite{soft_cosine2017}.
Hence, it improves the shortcomings of the traditional cosine similarity measure, 
which assumes the tokens of the vocabulary are completely independent of each other \cite{soft_cosine2014}.

According to \citeauthor{soft_cosine2014}, in order to model this additional information, more dimensions are added to the \ac{vsm}.
These dimensions can be obtained, for instance, by multiplying the mean of two features of one vector with the similarity between them \cite{soft_cosine2014}.
The similarity can be calculated by using Levenshtein distance, i.e. the number of operations necessary to convert one string into another, 
or using a dictionary of synonyms.

Since this approach no longer assumes that different words are independent of each other, 
the basis vectors which span the \ac{vsm} are no longer presumed orthogonal.
The formula for the soft cosine similarity is defined in \autoref{eq:soft-cosine-similarity} from \cite{soft_cosine2014}.
The similarity $s_{ij}$ between the $i$-th and $j$-th basis vector is obtained using a similarity measure, such as synonymy.

\begin{equation}
    soft\_cosine(a,b) = \frac{\sum_{i=1}^{N}\sum_{j=1}^{N}s_{ij}a_{i}b_{j}}{\sqrt{\sum_{i=1}^{N}\sum_{j=1}^{N}s_{ij}a_{i}a_{j}}\sqrt{\sum_{i=1}^{N}\sum_{j=1}^{N}s_{ij}b_{i}b_{j}}}
    \label{eq:soft-cosine-similarity}
\end{equation}

According to \citeauthor{soft_cosine2017}, the similarity between two texts is non-zero as soon as they share related words \cite{soft_cosine2017}.
If there is no similarity between different features, 
the soft cosine similarity from \autoref{eq:soft-cosine-similarity} is equal to the cosine similarity from \autoref{eq:cosine-similarity}.
%The time and space complexity of the soft cosine similarity is $O(N^2)$ \cite{soft_cosine2014}.
% In order to reduce the complexity, \citeauthor{soft_cosine2014} propose the usage of a sparse similarity matrix which only stores $s_{ij} > t$, 
% $t$ being a threshold.
\section{Embeddings}\label{sec:embeddings}

Usually, \ac{ml} techniques require textual inputs to be converted to embeddings \cite{SentRep2014}.
Embeddings are numerical representations of words, sentences or texts.
They can be used to present the textual data as real-valued vectors in a \ac{vsm}.
A \ac{vsm} is a $N$-dimensional space \cite{soft_cosine2014}.
%Each dimension of a \ac{vsm} corresponds to an index term, which is dependent on the embedding model.
%Every document embedding dimension explains the importance of the corresponding index term to the document.
\acp{vsm} are commonly used due to their conceptual simplicity and because spatial proximity correlates with semantic proximity 
\cite{tfidf2008, UniversalSentEnc2018, HfsentTrans2019, Top2Vec2020}.
Representations in a \ac{vsm} can improve the performance in \ac{nlp} tasks \cite{SkipGram2013}.
% According to \citeauthor{tfidf2008}, when representing text the first step is indexing, i.e.\ assigning indexing terms to the document.
% The second task is to assign weights to the terms that correspond to the importance of the term in the document.
% The weights assigned depend on the model.

The following section outlines the fundamentals of a selection of embeddings.
Let a corpus of documents be denoted $D= \left\{d_1, d_2, ..., d_M  \right\}$ 
and a sequence of terms $w_{ij}$ or so-called document $d_i = \left\{w_{i1}, w_{i2}, ..., w_{iV}  \right\}$, 
$V$ being the length of the vocabulary, 
i.e.\ set of distinct words \cite{clusteringDocs2020}, and $j \in [0, V]$.


\subsection{\acl*{tfidf}}\label{subsec:tfidf}

\ac{tfidf} provides a numerical representation of a word in a document \cite{clusteringDocs2020}.
Let a corpus of documents be denoted $D= \left\{d_1, d_2, ..., d_M  \right\}$, $M$ being the total number of documents in the corpus. 
Let a sequence of terms $w_{j} \in V$ be denoted a document $d_i = \left\{w_{1}, w_{2}, ...\right\}$, 
$V$ being the vocabulary, 
i.e.\ set of distinct words \cite{clusteringDocs2020}.

The \ac{tfidf} model considers the frequency $f_{w_{j}, d}$  of a word $w_{j}$ in a document $d$ and the frequency of a word in the whole corpus. 
The frequency $f_{w_{j}, d}$ is defined in \autoref{eq:tfidf_frequency}, $w'_j$ being the number of occurrences of $w_j$ in $d$.

\begin{align}
    f_{w_{j}, d} &= \frac{w'_{j}}{\sum_{k \in d} w'_k}\label{eq:tfidf_frequency}\\
    TFIDF(w_{j}, d, D) &= TF(w_{j}, d) \cdot IDF(w_{j}, D)\label{eq:tfidf_calculation}\\
    TF(w_{j}, d) &= f_{w_{j}, d}\label{eq:tf_calculation}\\
    IDF(w_{j}, D) &= \log_2\frac{M}{M_{j}}\label{eq:idf_calculation}
\end{align}

\ac{tfidf} is calculated using \autoref{eq:tfidf_calculation} from \cite{clusteringDocs2020}.
Each entry of a \ac{tfidf} embedding vector represents the \ac{tfidf} value of a word in a document.
Hence, the embedding vector is of the same length as the vocabulary of the corpus.
The \ac{tf} is determined utilizing \autoref{eq:tf_calculation}, 
whereas the \ac{idf} is computed by \autoref{eq:idf_calculation}, 
$M_{j}$ being the number of documents the term $w_{j}$ appears in.

\ac{idf} measures the importance of a term $w_{j}$ in the corpus of documents $D$
under the assumption that a term's importance to the data corpus is inversely proportional to its occurrence frequency \cite{tfidf2008}.
In other words: Terms which appear in many documents are not as important and thus, weighted less than document-specific terms. 
The calculation of \ac{tf} and \ac{idf} is visualized exemplary in \autoref{fig:tfidf-calculation}.


\begin{figure}[!htb] % htp = hier (h), top (t), oder auf einer eigenen Seite (p).
    \centering
    \includesvg[width=0.7\textwidth]{images/embeddings/tfidf/tfidf.svg}
    \caption[Exemplary calculation of \acs*{tf} and \acs*{idf} values]{
        Exemplary calculation of \acs*{tf} and \acs*{idf} for a document corpus $D$: 
        \acs*{tf} only considers the documents of interest while 
        \acs*{idf} incorporates the importance of the word with respect to $D$.
    }
    \label{fig:tfidf-calculation}
\end{figure}

%According to \citeauthor{tfidf2008}, the computation complexity of \ac{tfidf} embeddings is $O(V \cdot M)$.
\ac{tfidf} has several drawbacks \cite{clusteringDocs2020,tfidf2008}:
\begin{itemize}
    \item \ac{tfidf} does not consider semantic similarities between words.
    \item \ac{tfidf} does not take into account the order of words in a document.
    \item \ac{tfidf} often produces high dimensional representations which have to be postprocessed to reduce their dimensionality, e.g., by using \ac{pca}.
    %\item The embeddings are not derived from a mathematical model of term distribution and thus, are occasionally criticised as not well reasoned.
\end{itemize}

% TODO: advantages of tfidf?

\subsection{\ac{d2v}}\label{subsec:doc2vec}

Another term used for \ac{d2v} is \textit{Paragraph Vector} \cite{clusteringDocs2020, SentRep2014}.
\ac{d2v} addresses the problems of \ac{tfidf} by encoding texts as $N-$dimensional vectors learnt using the words' context \cite{clusteringDocs2020}.
Hence, it preserves semantic similarities between words and encodes linguistic regularities and patterns \cite{SkipGram2013}.
% According to \citeauthor{clusteringDocs2020} and \citeauthor{SentRep2014}, 
% \ac{d2v} learns continuous distributed vector representations for pieces of the text.
The model handles inputs of different dimensions and thus, tokens can be sentences, paragraphs or documents.

\ac{d2v} is an adaption of the \ac{w2v} model, which maps words into a \ac{vsm} \cite{clusteringDocs2020}.
Both approaches assume that words appearing in similar contexts are semantically similar. %\cite{clusteringDocs2020}.
The \ac{w2v} embedding is obtained using a shallow \ac{nn}, i.e. the \ac{nn} has only one hidden layer.
The embeddings are created by the hidden layer.
There are two approaches to designing the architecture of the \ac{nn}:
\begin{itemize}
    \item \ac{pvdm}: 
        Predicts a word given a context \cite{SentRep2014, WordRep2013}.
    \item \ac{pvdbow}: 
        Predicts the context given a word \cite{EmbDist2015, SkipGram2013, SentRep2014}.
\end{itemize}

% \begin{equation}
%     \frac{1}{T}\sum_{t=k}^{T-k}\text{log}(p(w_t | w_{t-k}, ..., w_{t+k}))
%     \label{eq:word2vec-cbow}
% \end{equation}

\begin{figure}%
    \centering
    \subfloat[\centering \ac{cbow} architecture cf. \cite{WordRep2013}.]
    {{\includesvg[width=7cm]{images/embeddings/doc2vec/CBOW.svg}}}%
    \qquad
    \subfloat[\centering \ac{pvdm} architecture cf. \cite{SentRep2014}.]
    {{\includesvg[width=6cm]{images/embeddings/doc2vec/PV-DM.svg} }}%
    \caption[\ac{cbow} and \ac{pvdm} architecture]{Both approaches predict the centre word using the context.
    \ac{pvdm} is an adaption of \ac{cbow} to work on a set of documents or paragraphs instead of words.
    }%
    \label{fig:pvdm}%
\end{figure}
 
% context to center word
\ac{pvdm} extends \ac{cbow} to work on a corpus of documents instead of on a set of words \cite{clusteringDocs2020}:
As usual, vectors representing the words are obtained using the \ac{cbow} model.
\ac{cbow} has a single-layer architecture \cite{glove2014}. %based on the inner product between two word vectors \cite{glove2014}.
%Given training words $ w_{1}, ..., w_{T}$, the objective of the \ac{w2v} model \ac{cbow} is to maximize the average log probability in 
%\autoref{eq:word2vec-cbow} from \cite{SentRep2014}.
The word vectors can be concatenated, averaged or summed up \cite{SentRep2014}.
Each document is mapped to a vector using an additional document-to-vector matrix.
Both document and word vectors are initialized randomly, but trained to convey meaning in terms of semantic differentiation.
In order to train the model, center words are predicted using the context.
The context consists of words within a sliding window and their document, respectively represented as vectors \cite{SentRep2014}.
The document vector is added to incorporate the document's topic and thus, acts like a memory \cite{SentRep2014, Top2Vec2020}.
\citeauthor{SentRep2014} concatenate the document vector to the word vectors.
The resulting vector is the prediction of the central word.
The approaches are displayed in \autoref{fig:pvdm}.

According to \citeauthor{SentRep2014}, both \ac{cbow} and \ac{pvdm} are trained using stochastic gradient descent and backpropagation.
They also state that the document vectors are unique, while the word vectors are shared across the whole corpus. 

\begin{figure}%
    \centering
    \subfloat[\centering Skip-gram architecture cf. \cite{WordRep2013}.]
    {{\includesvg[width=7cm]{images/embeddings/doc2vec/Skip-gram.svg}}}%
    \qquad
    \subfloat[\centering \ac{pvdbow} architecture cf. \cite{SentRep2014}.]
    {{\includesvg[width=4.5cm]{images/embeddings/doc2vec/PV-DBOW.svg} }}%
    \caption[Two \ac{pvdbow} architectures]{Both approaches predict the context.
    \ac{pvdbow} is an adaption of Skip-gram to work on a set of documents or paragraphs instead of words.
    }%
    \label{fig:pvdbow}%
\end{figure}
% center word to context
The \ac{pvdbow} approach is the adaption of the \ac{w2v} algorithm Skip-Gram and predicts the context 
given the document \cite{SentRep2014}.
The approaches are displayed in \autoref{fig:pvdbow}.
%According to \citeauthor{SentRep2014}, the 


% paper widerspricht sich
%According to \citeauthor{clusteringDocs2020}, the \ac{doc2vec} model's performance is influenced quality of the preprocessing.
%If for instance, the stemmer assigns words with different meaning to the same root, there is a degradation in performance.

\subsection{\acl*{use}}\label{subsec:univ-sent-encoder}

\citeauthor{UniversalSentEnc2018} have published their \acf{use} model on TensorFlow Hub.
They propose two architectures, one based on a Transformer and one based on a \ac{dan} \cite{UniversalSentEnc2018}.
Both models' input is a lowercase tokenized string.
Their output is a 512-dimensional vector.

% Transformer
The transformer model is more accurate and more complex than the \ac{dan} model \cite{UniversalSentEnc2018}.
The transformer's (self) attention is used to compute context-aware word embeddings, which consider both the word order and their semantic identity.
Since a sequence of word embeddings of a sentence produces embeddings of different dimensions, the approach postprocesses the word embeddings.
A sentence vector is obtained by computing the element-wise sum of the word embeddings 
and normalizing the result by dividing by the square root of the sentence length.

% DAN
The \ac{dan} model receives real-valued embeddings of words and bi-grams as input.
A bi-gram is a tupel of two subsequent words in a text \cite{nlp-book2009}, for instance, \textit{(red, wine), (wine, tastes), (tastes, good)}.
The embeddings can be obtained from the text strings using models such as the \ac{bow} model \cite{UniversalSentEnc-dan-input-emb}.
They are averaged and subsequently passed to a feedforward \ac{dnn} \cite{UniversalSentEnc2018}.
The architecture of the \ac{dan} model is depicted in \autoref{fig:use_dan}.

\begin{figure}[!htp] % htp = hier (h), top (t), oder auf einer eigenen Seite (p).
    \centering
    \includesvg[width=0.9\textwidth]{images/embeddings/universal_sentence_encoder/DAN.svg}
    \caption[Architecture of \acs*{use}]{Architecture of the \acs*{dan} model used for \acs*{use} based on the textual description from \cite{inferSent2018}.
    The input words and bi-grams $(w_1, w_2, ..., w_N)$ are embedded.
    The embeddings are averaged and subsequently passed to a feedforward \acs*{dnn}, which produces a 512-dimensional sentence embedding.
    }
    \label{fig:use_dan}
\end{figure}

% dataset
The models are trained on both unsupervised training data, e.g., Wikipedia, and a supervised training dataset, i.e.\ \ac{snli} \cite{UniversalSentEnc2018, HfsentTrans2019}.
The unsupervised training task is to predict the context given an input, i.e.\ Skip-Gram like tasks.
The supervised training task is classification \cite{UniversalSentEnc2018}.

% complexity
%The transformer model is more complex than the \ac{dan} model.
% The transformer model complexity is $O(n^2)$, whereas the \ac{dan} model complexity is $O(n)$, 
% $n$ being the number of words in the sentence \cite{UniversalSentEnc2018}.
% % memory usage
% The memory usage of both models is equivalent to their complexity.
% \citeauthor{UniversalSentEnc2018} state that \ac{dan}'s memory usage is dominated by the parameters used to store the embeddings of the uni- and bi-grams.
% Moreover, the transformer model only stores the uni-gram embeddings and thus, can require less memory than \ac{dan} for short sentences \cite{UniversalSentEnc2018}.

\subsection{\infersent{}}\label{subsec:inferSent}

% general
\infersent{} is a sentence embedding method trained in a supervised manner on the \ac{snli} dataset \cite{inferSent2018, HfsentTrans2019}.
The trained model is transferable to other tasks.
% Bi-LSTM as sentence encoder
\citeauthor{inferSent2018} compare multiple architectures in their work.
The \ac{bilstm} architecture with max pooling which was found to be the best option for the sentence encoder 
is depicted in \autoref{fig:infersent_bilstm} \cite{inferSent2018}.

\begin{figure}[!htb] % htp = hier (h), top (t), oder auf einer eigenen Seite (p).
    \centering
    \includesvg[width=0.9\textwidth]{images/embeddings/infersent/Infersent.svg}
    \caption[Architecture of \infersent{}]{Architecture of the \acs*{bilstm} model with max pooling used for \infersent{} cf. \cite{inferSent2018}.
    The input sentence $(w_1, w_2, ..., w_T)$ is read from both directions by a forward and a backward \acs*{lstm} 
    producing $\overrightarrow{h_t}$ and $\overleftarrow{h_t}$ respectively.
    After concatenating $\overrightarrow{h_t}$ and $\overleftarrow{h_t}$ to $h_t$, max pooling is applied.
    The output is a fixed-sized embedding.
    }
    \label{fig:infersent_bilstm}
\end{figure}

% LSTM
A \ac{lstm} is a \ac{rnn} that is capable of learning long-term dependencies.
\acp{rnn} have closed loops, i.e.\ feedback connections between the nodes \cite{rnn_book2001}.
In other words, 
a \ac{lstm} is able to remember information as a so-called \textit{state}.
Certain \ac{lstm} mechanisms control whether the current state is deleted, whether new data is saved and 
to what degree the current state contributes to the current input processed in the node.
Hence, \ac{lstm} nodes are not only influenced by former outputs but also by their state.
Since the \ac{lstm} computes different numbers of hidden vectors $h_t$ depending on the length of a sentence, 
a max pooling layer is applied to the hidden vectors which selects the maximum value for a patch of the hidden vectors.

% Bi-LSTM
According to \citeauthor{HfsentTrans2019}, \infersent{} consists of a single \ac{bilstm} layer \cite{HfsentTrans2019}.
Given a sentence $(w_1, w_2, ..., w_T)$ of $T$ words, the \ac{bilstm} architecture computes the hidden representations $h_t$ for each word $w_t$.
The hidden representation $h_t$ is the concatenation of the forward and backward hidden vectors $\overrightarrow{h_t}$ and $\overleftarrow{h_t}$.
$\overrightarrow{h_t}$ and $\overleftarrow{h_t}$ are produced by a forward and backward \ac{lstm} respectively.
Hence, the sentence is read from both directions and thus, considers past and future context.

\subsection{\acl*{sbert}}\label{subsec:hf-sent-ransformers}

\ac{sbert} is an enhancement of \ac{bert}.
% BERT
\ac{bert} is a pre-trained transformer network.
It predicts a target value, for i.e. classification or regression tasks, based on two input sentences \cite{HfsentTrans2019}.
The input sentences are separated by a special token \texttt{[SEP]}.
The base model applies multi-head attention over 12 transformer layers, whereas the large model applies multi-head attention over 24 transformer layers.
The final label is derived from a regression function, which receives the output of the $12^\text{th}$ or $24^\text{th}$ layer, respectively.
\citeauthor{HfsentTrans2019} state that \ac{bert} is not suitable for specific pair regression tasks, 
since the number of input sentence combinations is too big.
Another shortcoming of \ac{bert} is that it does not produce independent embeddings for single sentences.
Moreover, \citeauthor{HfsentTrans2019} found that common similarity measurements, for instance, the ones discussed in \autoref{sec:similarity-measurement}, 
do not perform well on sentence embeddings produced by \ac{bert} \cite{HfsentTrans2019}.

% SBERT
\ac{sbert} provides fixed-sized embeddings for single sentences \cite{HfsentTrans2019}.
It differs from \ac{bert} in terms of architecture, since it adds a pooling layer after the \ac{bert} model.
\citeauthor{HfsentTrans2019} compare different pooling strategies, such as using the output of the \texttt{CLS} (i.e. first) token, mean pooling and max pooling.
The architecture of a single \ac{sbert} network is depicted in \autoref{fig:sbert}.
In order to work with multiple input sentences at the same time, siamese and triplet network architectures, 
i.e. multiple \ac{bert} networks with tied weights, are constructed.
To perform classification or inference tasks layers are added on top of the \ac{sbert} network.
% training corpus
\ac{sbert} is trained on the \ac{snli} dataset.

\begin{figure}[!htb] % htp = hier (h), top (t), oder auf einer eigenen Seite (p).
    \centering
    \includesvg[width=0.7\textwidth]{images/embeddings/SBERT/SBERT.svg}
    \caption[Architecture of \ac{sbert}]{Architecture of \ac{sbert} cf. \cite{HfsentTrans2019}.
    \ac{bert} is extended by a pooling layer.
    The input is a sentence and the output is a fixed-sized embedding.
    }
    \label{fig:sbert}
\end{figure}

% performance
According to \citeauthor{HfsentTrans2019}, \ac{sbert} outperforms \infersent{} and \ac{use} on Semantic Textual Similarity tasks 
and on SentEval, which is an evaluation toolkit for sentence embeddings \cite{HfsentTrans2019}.
Moreover, due to \ac{sbert}'s transformer architecture, it is more computationally efficient than \ac{use} on \acp{gpu}.
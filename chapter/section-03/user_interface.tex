\section{User Interface}\label{sec:ui}

\subsection{Backend}\label{subsec:backend}

% this work: endpoints
In this work, only the \texttt{GET} method is used.
There are multiple endpoints, which are used to retrieve data from the server:

\begin{itemize}
    \item \label{pt:docs}Documents: 
        Returns a list of documents, which best match the query.
        The query can be of type \texttt{match\_all}, which returns all documents in the database, 
        or a fuzzy full-text query, 
        or a \ac{knn} query on a certain field of the database.
        Moreover, the number and start index of the results returned can be specified.

    \item \label{pt:doc}Document: 
        Returns the document with the specified \texttt{id}.

    \item \label{pt:pdf}\ac{pdf}: 
        Returns the path to a \ac{pdf} file.
        In order to access the path information a query for a document with the specified \texttt{id} is performed.
    
    \item \label{pt:wordcloud}WordCloud: 
        Returns the bytes of a WordCloud image. 
        Depending on additional parameters, the WordCloud is either generated from one document or 
        the most similar documents to the query field, identified by \ac{knn}.

    \item \label{pt:termfrequency}\textcolor{red}{Term Frequency}
\end{itemize}

In order to test the endpoints during development, swagger documentation for every endpoint is provided.





\subsection{Frontend}\label{subsec:frontend}
angular
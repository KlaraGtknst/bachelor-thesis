\section{User Interface}\label{sec:ui}

\subsection{Backend}\label{subsec:backend}

% introduction
The framework used for the backend is \flask{}.
According to \citeauthor{flask_book2015}, \flask{} is one of the most popular Python web frameworks.
It provides powerful libraries for tasks such as routing, templating, and \ac{http} request parsing \cite{flask_book2015}.

% initialization
All requests received from clients are passed to an instance of the \flask{} application \cite{flask_book2018}.
% code example

% routing
Clients send requests to the web server, which passes them to the \flask{} application instance.
The queries are then routed to the corresponding functions.
Routing is the process of mapping \ac{url} paths to functions \cite{flask_book2018}.
To define a route, the \texttt{route} decorator is used \cite{flask_book2018}.
% code example
\acs{url} can contain dynamic components, which are enclosed in \texttt{<>} angle brackets.
The values of these components are passed to the function as arguments \cite{flask_book2018}.
By default, dynamic components are of type \texttt{string}.
However, other types including \texttt{int} and \texttt{float} are supported \cite{flask_book2018}.

% dev server
During development, the \flask{} application can be run using \texttt{flask run} to start the built-in development web server \cite{flask_book2018}.
By enabling debug mode, the server automatically reloads the application when changes are detected \cite{flask_book2018}.

\cite{mvc_flask2019}
\cite{flask2015}
\cite{flask_book2015}
\cite{flask_book2018}
\cite{flask2018}

\subsection{Frontend}\label{subsec:frontend}
angular
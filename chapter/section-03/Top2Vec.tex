\subsection{Top2Vec}\label{subsec:top2vec}

The approach \topTwovec{} was proprosed by \citeauthor{Top2Vec2020} \cite{Top2Vec2020}.
It addresses several problems of state-of-the-art topic modelling approaches, such as \ac{lda}.
Opposed to \ac{lda}, \topTwovec{} does not require the user to specify the number of topics $k$, 
i.e. it does not discretize the topic space into $k$ topics, 
and it does not require stop word removal or lemmatization.
Moreover, it considers the semantic meaning of words unlike \ac{lda}.
In contrast to \ac{lda}, \topTwovec{} only associates one topic with a document.

% word2vec and doc2vec
\topTwovec{} is based on \ac{w2v} and \ac{d2v}.
The documents are embedded using \ac{pvdbow}.
The learning task is to predict the document a word came from \cite{Top2Vec2020}.
Similar to \ac{d2v}, \topTwovec{} not only embeds words and documents in the same feature space but also topics.
\citeauthor{Top2Vec2020} regards each point in the \ac{vsm} as a topic, described by its nearest words.

% topics & clustering
\citeauthor{Top2Vec2020} states that topics are continuous and can be described by different sets of words \cite{Top2Vec2020}.
Hence, topic modelling is defined as the task of finding sets of informative words that describe a document.
Documents in dense areas of the topic space are considered to be about the same topic.
The density-based clustering algorithm \ac{hdbscan} is used to find these dense areas.
The topic vector is denoted the centroid or average of the document vectors belonging to the topic.
\citeauthor{Top2Vec2020} has compared several topic vector definitions and found the arithmetic mean to be the best option \cite{Top2Vec2020}.
The number of topics is derived from the number of dense areas.
It is possible to merge topics to hierarchically reduce the number of topics found to any number of topics smaller than the number initially found.

% dimensionality reduction
In order to find topics, clusters of documents, i.e. dense areas, need to be identified.
The clustering algorithm \ac{hdbscan} is used to find these dense areas.
Since \ac{hdbscan} has difficulties finding dense clusters in high-dimensional data, 
the dimensionality reduction method \ac{umap} is applied \cite{Top2Vec2020}.
The steps of the topic modelling procedure \topTwovec{} are depicted in \autoref{fig:top2vec}.

\begin{figure}[htp] % htp = hier (h), top (t), oder auf einer eigenen Seite (p).
    \centering
    \includesvg[width=0.7\textwidth]{images/topic_modelling/Top2Vec.svg}
    \caption{Procedure of topic modelling using \topTwovec{}.}
    \label{fig:top2vec}
\end{figure}
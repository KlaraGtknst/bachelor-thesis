\subsection{Top2Vec}\label{subsec:top2vec}

The approach \topTwovec{} addresses several problems of state-of-the-art topic modelling approaches \cite{Top2Vec2020}.
Opposed to \ac{lda}, \topTwovec{} does not require the user to specify the number of topics $k$, 
i.e. it does not discretize the topic space into $k$ topics, 
and it does not require stop word removal or lemmatization.
Moreover, it considers the semantic meaning of words unlike \ac{lda}.
In contrast to \ac{lda}, \topTwovec{} only associates one topic with a document.

\begin{figure}%
    \centering
    \subfloat[\centering Skip-gram architecture cf. \cite{Topic2Vec2015}.]
    {{\includesvg[width=5.5cm]{images/topic_modelling/Skip-gram_Top2Vec.svg}}}%
    \qquad
    \subfloat[\centering \ac{cbow} architecture cf. \cite{Topic2Vec2015}.]
    {{\includesvg[width=5.5cm]{images/topic_modelling/CBOW_Top2Vec.svg} }}%
    \caption[Two learning architectures of \topTwovec{}]{Both learning architectures of \topTwovec{}.
    $w(t-2), w(t-1), w(t+1), w(t+2)$ are the context words of the centre word $w(t)$ of topic $z(t)$.
    }%
    \label{fig:top2vec_architectures}%
\end{figure}

% word2vec and doc2vec
\topTwovec{} is based on \ac{w2v} and \ac{d2v}.
The documents are embedded using \ac{pvdbow}.
The two learning architectures \ac{cbow} and Skip-gram from \ac{w2v} are adapted to train the model as depicted in \autoref{fig:top2vec_architectures} \cite{Topic2Vec2015}.
The Skip-Gram learning task is to predict the document a word came from \cite{Top2Vec2020, Topic2Vec2015}.
Similar to \ac{d2v}, \topTwovec{} not only embeds words and documents in the same feature space but also topics \cite{Top2Vec2020, Topic2Vec2015}.
The similarity between embeddings can be measured using the cosine similarity function \cite{Topic2Vec2015}.
\citeauthor{Top2Vec2020} regards each point in the \ac{vsm} as a topic, described by its nearest words.

% topics, dimensionality reduction & clustering
\citeauthor{Top2Vec2020} states that topics are continuous and can be described by different sets of words \cite{Top2Vec2020}.
Hence, topic modelling is defined as the task of finding sets of informative words that describe a document.
Documents in dense areas of the topic space are considered to be about the same topic.
The density-based clustering algorithm \ac{hdbscan} is used to find these dense areas.
Since \ac{hdbscan} has difficulties finding dense clusters in high-dimensional data, 
the dimensionality reduction method \ac{umap} is applied \cite{Top2Vec2020}.
The steps of the topic modelling procedure \topTwovec{} are depicted in \autoref{fig:top2vec}.

A topic vector is denoted as the centroid or average of the document vectors belonging to a certain topic.
%\citeauthor{Top2Vec2020} compared several topic vector definitions and found the arithmetic mean to be the best option \cite{Top2Vec2020}.
The number of topics is derived from the number of dense areas.
It is possible to merge topics to hierarchically reduce the number of topics found to any number smaller than the number of topics initially found.

% complexity
According to \citeauthor{Topic2Vec2015}, the complexity of \topTwovec{} is linear with the size of the dataset \cite{Topic2Vec2015}.

\begin{figure}[htp] % htp = hier (h), top (t), oder auf einer eigenen Seite (p).
    \centering
    \includesvg[width=0.7\textwidth]{images/topic_modelling/Top2Vec.svg}
    \caption{Procedure of topic modelling using \topTwovec{}.}
    \label{fig:top2vec}
\end{figure}
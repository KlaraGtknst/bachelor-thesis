\section{Embeddings}\label{sec:embeddings}

Usually, \ac{ml} techniques embeddings, such as K-Means, require the text input data to be converted to embeddings \cite{SentRep2014}.
Embeddings are numerical representations of words, sentences or texts.
They can be used to present the textual data as real-valued vectors in a \ac{vsm}.
\acp{vsm} are commonly used due to their conceptual simplicity and because spatial proximity serves as a metaphor for semantic proximity 
\cite{tfidf2008, UniversalSentEnc2018, HfsentTrans2019, Top2Vec2020}.
Representations in a vector space can improve the performance in \ac{nlp} tasks \cite{SkipGram2013}.
According to \citeauthor{tfidf2008}, when representing text the first step is indexing, i.e. assigning indexing terms to the document.
The second task is to assign weights to the terms which correspond to the importance of the term in the document.
The weights assigned depend on the method and the assumptions of the model chosen to carry out the assignments.

The following section outlines a selection of embeddings.
Let a corpus of documents be denoted $D= \left\{d_1, d_2, ..., d_M  \right\}$, the number of documents in the dataset $M = \left\| D \right\|$,
a sequence of terms $w_{ij}$ or so-called document $d_i = \left\{w_{i1}, w_{i2}, ..., d_{iV}  \right\}$, $V$ being the length of the vocabulary, 
i.e. set of distinct words, of the corpus of documents \cite{clusteringDocs2020}.


\subsection{\ac{tfidf}}\label{subsec:tfidf}

\ac{tfidf} provides a numerical representation of a word in a document \cite{clusteringDocs2020}.
It considers the frequency of a word in a document and the frequency of a word in the whole corpus. 

\ac{tfidf} is calculated as displayed in \autoref{eq:tfidf-formula} from \cite{clusteringDocs2020} and exemplatory in \autoref{fig:tfidf-calculation}.
\ac{tf} is computed using $TF(w_{ij}, d_i) = f_{w_{ij}, d_i}$, whereas the \ac{idf} is computed using $IDF(w_{ij}, D) = \log_2\frac{M}{M_{ij}}$, 
$M_{ij}$ being the number of documents the term $w_{ij}$ appears in.
\ac{idf} measures the importance of a term $w_{ij}$ in the corpus of documents $D$.
The underlying assumption of \ac{idf} is that a term's importance to the data corpus is inversely proportional to its occurrence frequency \cite{tfidf2008}.
In other words: Terms which appear in many documents are not as important and thus, weighted less than document-specific terms. 

\begin{equation}
    TFIDF(w_{ij}, d_i, D) = TF(w_{ij}, d_i) \cdot IDF(w_{ij}, D)
    \label{eq:tfidf-formula}
\end{equation}


\begin{figure}[htp] % htp = hier (h), top (t), oder auf einer eigenen Seite (p).
    \centering
    \includesvg[width=0.7\textwidth]{images/embeddings/tfidf/tfidf.svg}
    \caption{
        Example of calculation of \ac{tfidf} parts: 
        \ac{tf} only considers the documents of interest while 
        \ac{idf} incorporates the importance of the word with respect to the whole dataset.
    }
    \label{fig:tfidf-calculation}
\end{figure}

\textcolor{red}{TODO: svg about tfidf}

According to \citeauthor{tfidf2008}, the computation complexity of \ac{tfidf} embeddings is $O(V \cdot M)$
The \ac{tfidf} more has several drawbacks \cite{clusteringDocs2020,tfidf2008}:
\begin{itemize}
    \item \ac{tfidf} does not consider semantic similarities between words.
    \item \ac{tfidf} does not consider the order of words in a document.
    \item \ac{tfidf} often produces high dimensional representations which have to be postprocessed to reduce their dimensionality, e.g., using \ac{pca}.
    \item The embeddings are not derived from a mathematical model of term distribution and may be criticised as not well reasoned.
\end{itemize}

\subsection{\ac{d2v}}\label{subsec:doc2vec}

Another term used for \ac{d2v} is \textit{Paragraph Vector} \cite{clusteringDocs2020}.
\ac{d2v} addresses the problems of \ac{tfidf} by encoding texts as $n-$dimensional vectors learnt using the words' context \cite{clusteringDocs2020}.
Hence, it preserves semantic similarities between words.
According to \citeauthor{clusteringDocs2020}, \ac{d2v} learns continuous distributed vector representations for pieces of the text.
The model handles inputs of different dimensions.

\ac{d2v} is an adaption of the \ac{w2v} model, which maps words into a \ac{vsm} under consideration of their semantic similarities \cite{clusteringDocs2020}.
The underlying hypothesis of both approaches is that words appearing in similar contexts are semantically similar \cite{clusteringDocs2020}.
The \ac{w2v} embedding is obtained using a \ac{nn} \cite{clusteringDocs2020}.
The \ac{nn} is shallow, i.e. has only one hidden layer.
This hidden layer creates the embedding of input data.
There are two approach as to how design the architecture of the \ac{nn}:
\begin{itemize}
    \item \ac{cbow}: 
        Predicts a word given a context
    \item Skip-Gram: 
        Predicts the context given a word.
\end{itemize}

The \ac{pvdm} extends the \ac{cbow} to work on corpus of documents instead of set of words \cite{clusteringDocs2020}:
As usual, vectors representing the words are obtained using the \ac{cbow} model.
Each document is mapped to a vector using an additional document-to-vector matrix.
The document vector is concatenated to the word vectors.
The resulting vector is used to predict the central word.
\textcolor{red}{prediction, loss function? B. in paper}

% paper widerspricht sich
%According to \citeauthor{clusteringDocs2020}, the \ac{doc2vec} model's performance is influenced quality of the preprocessing.
%If for instance, the stemmer assigns words with different meaning to the same root, there is a degradation in performance.


\cite{SentRep2014}
two flavor of doc2vec: PV-DM and PV-DBOW (https://thinkinfi.com/simple-doc2vec-explained/)
\cite{SkipGram2013}

\subsection{\ac{use}}\label{subsec:univ-sent-encoder}

\citeauthor{UniversalSentEnc2018} have published their \ac{use} models on TF Hub.
They propose two models, one based on a Transformer architecture and one based on a \ac{dan}.
Both models' input is a lowercase tokenized string.
Their output is a 512-dimensional vector.

% Transformer
The transformer model is more accurate and more complex than the \ac{dan} model \cite{UniversalSentEnc2018}.
The transformer's (self) attention is used to compute context-aware word embeddings, which consider both the word order and their semantic identity.
Since the sequence of word embeddings of a sentence would produce embeddings of different dimensions, the approach postprocesses the word embeddings.
A sentence vector is obtained by computing the element-wise sum of the word embeddings 
and normalizing the result by dividing by the square root of the sentence length.

% DAN
The \ac{dan} model receives embeddings of words and bi-grams as input.
The strings can be converted to input vectors using models such as the bag of words model \cite{UniversalSentEnc-dan-input-emb}.
The embeddings are averaged and subsequently passed to a feedforward \ac{dnn} \cite{UniversalSentEnc2018}.

% dataset
The models are trained on both unsupervised training data, e.g., Wikipedia, and the supervised training dataset \ac{snli} \cite{UniversalSentEnc2018, HfsentTrans2019}.

% complexity
The transformer model is more complex than the \ac{dan} model.
More specifically, the transformer model complexity is $O(n^2)$, whereas the \ac{dan} model complexity is $O(n)$, 
$n$ being the number of words in the sentence \cite{UniversalSentEnc2018}.

% memory usage
The memory usage of both models is equivalent to their complexity.
\citeauthor{UniversalSentEnc2018} state that \ac{dan}'s memory usage is dominated by the parameters used to store the embeddings of the uni- and bi-grams.
Moreover, the transformer model only stores the uni-gram embeddings and thus, can require less memory than \ac{dan} for short sentences \cite{UniversalSentEnc2018}.

\subsection{\infersent{}}\label{subsec:inferSent}

% general
\infersent{} is a sentence embedding method trained in a supervised manner on the \ac{snli} dataset \cite{inferSent2018, HfsentTrans2019}.
The trained model is transferable to other tasks.

% training
The \ac{snli} dataset contains a huge data corpus of English sentence pairs.
The sentence pairs are labelled with one of three categories: \textit{entailment}, \textit{contradiction} or \textit{neutral}.
This dataset is used because it captures \ac{nli} and thus, enables learning sentence semantics.
To train the model, a shared sentence encoder encodes both the premise and the hypothesis to their vector representations $u$ and $v$.
In order to extract information about the relation of $u$ and $v$, three matching methods are applied:

\begin{itemize}
    \item $(u,v)$: Concatenation of the two vectors.
    \item $u \cdot v$: Element-wise product.
    \item $|u - v|$: Element-wise difference of the two vectors.
\end{itemize}

The results of the matching methods are concatenated (cf. \cite{HfsentTrans2019}).
The resulting vector is then fed into a three-class classifier.
The classifier consists of multiple fully connected layers and a softmax layer \cite{inferSent2018}.

% Bi-LSTM as sentence encoder
\citeauthor{inferSent2018} have compared multiple architectures in their work.
The \ac{bilstm} architecture with max pooling has been found the best option for the sentence encoder \cite{inferSent2018}.
According to \citeauthor{HfsentTrans2019}, it is a single siamese \ac{bilstm} layer \cite{HfsentTrans2019}.
Given a sentence $(w_1, w_2, ..., w_T)$ of $T$ words, the \ac{bilstm} architecture computes the hidden representations $h_t$ for each word $w_t$.
The hidden representation $h_t$ is the concatenation of the forward and backward hidden vectors $\overrightarrow{h_t}$ and $\overleftarrow{h_t}$.
$\overrightarrow{h_t}$ and $\overleftarrow{h_t}$ are produced by a forward and backward \ac{lstm} respectively.
Hence, the sentence is read from both directions and thus, considers past and future context.

% LSTM
A \ac{lstm} is a \ac{rnn} that is able to learn long-term dependencies.
In other words: 
A \ac{lstm} is able to remember information as a so-called \textit{state}.
Certain \ac{lstm} mechanisms control whether the current state is deleted, whether new data is saved and 
to what degree the current state contributes to the current input processed in the node.
Hence, \ac{lstm} nodes are not only influenced by the former output but also by their state.

Since the \ac{lstm} computes different numbers of hidden vectors $h_t$ depending on the length of a sentence, a max pooling layer is applied to the hidden vectors.
The max pooling layer selects the maximum value for each dimension of the hidden vectors.

%LSTM: The sentence is represented by the last hidden vector $h_T$.

\subsection{Hugging face's \acs{sbert}}\label{subsec:hf-sent-ransformers}

\ac{sbert} is an enhancement of \ac{bert}.
% BERT
\ac{bert} is a pre-trained transformer network.
It predicts a target value, for i.e. classification or regression tasks, based on two input sentences \cite{HfsentTrans2019}.
The input sentences are separated by a special token \texttt{[SEP]}.
The base-model applies multi-head attention over 12 transformer layers, whereas the large model applies multi-head attention over 24 transformer layers.
The final label is derived from a regression function, which receives the output of the 12th or 24th layer, respectively.
\citeauthor{HfsentTrans2019} state that \ac{bert} is not suitable for specific pair regression tasks, 
since the number of input sentence combinations is too big.
Another shortcoming of \ac{bert} is that it does not produce independent embeddings for single sentences.
More \citeauthor{HfsentTrans2019} found that common similarity measurements, for instance, the ones discussed in \autoref{sec:similarity-measurement}, 
do not perform well on the representations of sentences in a \ac{vsm} produced by \ac{bert} \cite{HfsentTrans2019}.

% SBERT
\ac{sbert} is a modification of \ac{bert} that provides fixed-sized embeddings for single sentences \cite{HfsentTrans2019}.
It consists of a siamese and triplet network architecture.
It differs from \ac{bert} in terms of architecture, since it adds a pooling layer after the \ac{bert} model.
The pooling strategies compared by \citeauthor{HfsentTrans2019} are using the output of the first/ \texttt{CLS} token, mean pooling and max pooling.

% training corpus
\ac{sbert} is trained on the \ac{snli} dataset.

% performance
According to \citeauthor{HfsentTrans2019}, \ac{sbert} outperforms \infersent{} and \ac{use}.
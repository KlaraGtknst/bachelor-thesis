\section{Embeddings}\label{sec:embeddings}

Embeddings are numerical representations of words, sentences or texts.
They can be used to present the textual data as vectors in a \ac{vsm}.
\acp{vsm} are commonly used due to their conceptual simplicity and because spatial proximity serves as metapher for semantic proximity \cite{tfidf2008}.
According to \citeauthor{tfidf2008}, when representing text the first step is indexing, i.e. assigning indexing terms to the document.
The second task is to assign weights to the terms which correspond to the importance of the term in the document.
The weights assigned depend on the method and the assumptions of the model chosen to carry out the assignments.

The following section outlines a selection of embeddings.
Let a corpus of documents be denoted $D= \left\{d_1, d_2, ..., d_M  \right\}$, the number of documents in the dataset $M = \left\| D \right\|$,
a sequence of terms $w_{ij}$ or so-called document $d_i = \left\{w_{i1}, w_{i2}, ..., d_{iV}  \right\}$, $V$ being the length of the vocabulary, 
i.e. set of distinct words, of the corpus of documents \cite{clusteringDocs2020}.

\cite{WordRep2013}
\cite{SentRep2014}

\textcolor{red}{Skizze von Pipeline für jedes Embedding, welche zeigt, wie die Daten vorverarbeitet (stemming etc.) werden/ was das Model selber macht.}

%\subsection{\ac{cbow}}\label{subsec:bag-of-words}


\subsection{\ac{tfidf}}\label{subsec:tfidf}

\ac{tfidf} provides a numerical representation of a word in a document \cite{clusteringDocs2020}.
It considers the frequency of a word in a document and the frequency of a word in the whole corpus. 

\ac{tfidf} is calculated as displayed in \autoref{eq:tfidf-formula} from \cite{clusteringDocs2020} and exemplatory in \autoref{fig:tfidf-calculation}.
\ac{tf} is computed using $TF(w_{ij}, d_i) = f_{w_{ij}, d_i}$, whereas the \ac{idf} is computed using $IDF(w_{ij}, D) = \log_2\frac{M}{M_{ij}}$, 
$M_{ij}$ being the number of documents the term $w_{ij}$ appears in.
\ac{idf} measures the importance of a term $w_{ij}$ in the corpus of documents $D$.
The underlying assumption of \ac{idf} is that a term's importance to the data corpus is inversely proportional to its occurrence frequency \cite{tfidf2008}.
In other words: Terms which appear in many documents are not as important and thus, weighted less than document-specific terms. 

\begin{equation}
    TFIDF(w_{ij}, d_i, D) = TF(w_{ij}, d_i) \cdot IDF(w_{ij}, D)
    \label{eq:tfidf-formula}
\end{equation}


\begin{figure}[htp] % htp = hier (h), top (t), oder auf einer eigenen Seite (p).
    \centering
    \includesvg[width=0.7\textwidth]{images/embeddings/tfidf/tfidf.svg}
    \caption{
        Example of calculation of \ac{tfidf} parts: 
        \ac{tf} only considers the documents of interest while 
        \ac{idf} incorporates the importance of the word with respect to the whole dataset.
    }
    \label{fig:tfidf-calculation}
\end{figure}

\textcolor{red}{TODO: svg about tfidf}

According to \citeauthor{tfidf2008}, the computation complexity of \ac{tfidf} embeddings is $O(V \cdot M)$
The \ac{tfidf} more has several drawbacks \cite{clusteringDocs2020,tfidf2008}:
\begin{itemize}
    \item \ac{tfidf} does not consider semantic similarities between words.
    \item \ac{tfidf} does not consider the order of words in a document.
    \item \ac{tfidf} often produces high dimensional representations which have to be postprocessed to reduce their dimensionality, e.g., using \ac{pca}.
    \item The embeddings are not derived from a mathematical model of term distribution and may be criticised as not well reasoned.
\end{itemize}

\subsection{\ac{d2v}}\label{subsec:doc2vec}

Another term used for \ac{d2v} is \textit{Paragraph Vector} \cite{clusteringDocs2020}.
\ac{d2v} addresses the problems of \ac{tfidf} by encoding texts as $n-$dimensional vectors learnt using the words' context \cite{clusteringDocs2020}.
Hence, it preserves semantic similarities between words.
According to \citeauthor{clusteringDocs2020}, \ac{d2v} learns continuous distributed vector representations for pieces of the text.
The model handles inputs of different dimensions.

\ac{d2v} is an adaption of the \ac{w2v} model, which maps words into a \ac{vsm} under consideration of their semantic similarities \cite{clusteringDocs2020}.
The underlying hypothesis of both approaches is that words appearing in similar contexts are semantically similar \cite{clusteringDocs2020}.
The \ac{w2v} embedding is obtained using a \ac{nn} \cite{clusteringDocs2020}.
The \ac{nn} is shallow, i.e. has only one hidden layer.
This hidden layer creates the embedding of input data.
There are two approach as to how design the architecture of the \ac{nn}:
\begin{itemize}
    \item \ac{cbow}: 
        Predicts a word given a context
    \item Skip-Gram: 
        Predicts the context given a word.
\end{itemize}

The \ac{pvdm} extends the \ac{cbow} to work on corpus of documents instead of set of words \cite{clusteringDocs2020}:
As usual, vectors representing the words are obtained using the \ac{cbow} model.
Each document is mapped to a vector using an additional document-to-vector matrix.
The document vector is concatenated to the word vectors.
The resulting vector is used to predict the central word.
\textcolor{red}{prediction, loss function? B. in paper}

% paper widerspricht sich
%According to \citeauthor{clusteringDocs2020}, the \ac{doc2vec} model's performance is influenced quality of the preprocessing.
%If for instance, the stemmer assigns words with different meaning to the same root, there is a degradation in performance.


\cite{SentRep2014}
two flavor of doc2vec: PV-DM and PV-DBOW (https://thinkinfi.com/simple-doc2vec-explained/)
\cite{SkipGram2013}

%\subsection{\ac{w2v}}\label{subsec:word2vec}



\subsection{Universal sentence encoder}\label{subsec:univ-sent-encoder}
\ac{use}
\cite{UniversalSentEnc2018}

\subsection{\infersent{}}\label{subsec:inferSent}
\infersent{}
\cite{inferSent2018}

\subsection{Hugging face's sentence Transformers}\label{subsec:hf-sent-ransformers}
\cite{HfsentTrans2019}
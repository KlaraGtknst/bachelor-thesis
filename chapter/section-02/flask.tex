\section{\flask{}}\label{sec:BE_flask}

% introduction
The framework used for the backend is \flask{}.
\flask{} is open source and written in Python by Armin Ronancher in 2004 \cite{flask2015, mvc_flask2019}.
According to \citeauthor{flask_book2015} and \citeauthor{mvc_flask2019}, \flask{} is one of the most popular Python web frameworks.
It provides powerful libraries for core functionality such as routing, templating, and \ac{http} request parsing \cite{flask_book2015}.
It is extensible and thus, can be extended with additional plugins without affecting the internal structure of the existing system \cite{flask2015}.

% technical details
\flask{} uses the Jinja Template Engine for template files including \ac{html} pages,
whereas static files such as \ac{css} files are handeled using the Werkzeug WSGI toolkit \cite{flask2015}.
According to \citeauthor{flask2015}, Jinja is modeled after the Django template system.
Werkzeug implements, for instance, requests and response objects \cite{mvc_flask2019}.

% initialization
All requests received from clients are passed to an instance of the \flask{} application \cite{flask_book2018}.
Hence, the first step is to create an instance of the \flask{} class, such as done in \lst{lst:flask_app_init}.

\begin{listing}[htp]
    \begin{minted}{python3}
    app = Flask(__name__)
    \end{minted}
    \caption{Initialization of \flask{} application instance.
    }
    \label{lst:flask_app_init}
\end{listing}

% routing
Clients send requests to the web server, which passes them to the \flask{} application instance.
The queries are then routed to the corresponding functions.
Routing is the process of mapping \ac{url} paths to functions \cite{flask_book2018}.
To define a route, the \texttt{route} decorator is used as displayed in \lst{lst:flask_routing}.

\begin{listing}[htp]
    \begin{minted}{python3}
    @api.route('/documents/<id>', endpoint='document')
    class Document(Resource):
        def get(self, id):
            elastic_search_client = Elasticsearch(CLIENT_ADDR)
            return query_database.get_doc_meta_data(elastic_search_client, 
                doc_id=id)
    \end{minted}
    \caption{Exemplartary definition of a function to display routing with \flask{}.
    The \texttt{route} decorator is used to define the \ac{url} path.
    }
    \label{lst:flask_routing}
\end{listing}

\acs{url} can contain dynamic components, which are enclosed in \texttt{<>} angle brackets.
The values of these components are passed to the function as arguments \cite{flask_book2018}.
By default, dynamic components are of type \texttt{string}.
However, other types including \texttt{int} and \texttt{float} are supported \cite{flask_book2018}.

% dev server
During development, the \flask{} application can be run using \texttt{flask run} to start the built-in development web server \cite{flask_book2018}.
By enabling debug mode, the server automatically reloads the application when changes are detected \cite{flask_book2018}.

% endpoints
An endpoint is a class with certain methods, which can be accessed using \ac{http} requests.
Every endpoint can have a \texttt{GET}, \texttt{PUT} and a \texttt{DELETE} method \cite{flask2018}.
The \texttt{GET} method is used to retrieve data from the server, whereas the other methods are used to either insert or delete data.

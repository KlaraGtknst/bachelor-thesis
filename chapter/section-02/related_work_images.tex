% eigenfaces
This thesis aims to encode visual information as low-dimensional real-valued vectors.
Since the domain of face recognition deals with the task of deriving meaningful information from high dimensional data,
the \eigenfaces{} approach is adapted to document images in this work.
The task of finding similar images of faces is transferred to finding similar document images.
\eigenfaces{} projects face images into a lower-dimensional feature space which best encodes the variation among the faces \cite{eigenfaces1991}.
Since \citeyear{eigenfaces1991} this technique has been covered in a lot of papers 
\cite{eigenfaces1991, eigenfaces1997, eigenfaces2013, face-recognition2008, face-recognition2020, face-recognition2021}.
% clustering
The visual information of the document images should be used to cluster them.
\citeauthor{OPTICS1999} introduce a clustering algorithm called \ac{optics} which seems to be suitable for this task 
since it does not return an explicit clustering but a clustering structure \cite{OPTICS1999}.
Moreover, \citeauthor{OPTICS1999} state that \ac{optics} is a method for database mining.
% \ac{dbscan} and \ac{optics} are related.
% Hence, the employment of the clustering algorithm \ac{dbscan} on \ac{d2v} embeddings \cite{clusteringDocs2020}.
Other researchers, for instance, \citeauthor{OPTICS_kMeans_2016}, compare related clustering algorithms including K-Means, \ac{dbscan} and \ac{optics}.
They state that \ac{optics} overcomes \ac{dbscan}'s difficulties and K-Means limitations \cite{OPTICS_kMeans_2016}.
\citeauthor{OPTICS2013}, \citeauthor{OPTICS2014} and \citeauthor{OPTICS2016} propose \ac{optics} extensions, 
i.e.\ for spatially and temporally evolving data or a parallel version \cite{OPTICS2013, OPTICS2014, OPTICS2016}.

% fundamentals
Since working with huge amounts of textual data is a prevalent challenge in text-mining tasks,
\citeauthor{InformationRetrieval1999} have published a survey on \ac{ir} from documents in which
they explore methods to extract information from different formats, including texts and images \cite{InformationRetrieval1999}.
\citeauthor{nlp-book2009} published a book covering the \ac{nlp} fundamentals, including preprocessing techniques, using Python \cite{nlp-book2009}.

% models/ embeddings
In order to assess similarity among items of a corpus of textual data, documents are often represented in terms of real-valued vectors.
\citeauthor{WordRep2013}'s work covers continuous vector representations of words from very large datasets \cite{WordRep2013}.
Specific models used in this work include 
the \ac{d2v} model, the \ac{sbert} model \cite{HfsentTrans2019}, the \infersent{} model \cite{inferSent2018} and the \ac{use} model \cite{UniversalSentEnc2018}.
Both \citeauthor{clusteringDocs2020} and \citeauthor{SentRep2014} evaluate the \ac{d2v} model, 
arguing that it overcomes several of \ac{tfidf}'s shortcomings \cite{clusteringDocs2020} and that it is superior to \ac{bow} models \cite{SentRep2014}.
\citeauthor{UniversalSentEnc2018} compare the strengths of different \ac{use} architectures \cite{UniversalSentEnc2018}.

% time and dynamic
In contrast to the assumption of this thesis that the prevalent topics are static, in reality, topics may be dynamic or change over time.
Not only \citeauthor{topic_modeling2015}, but also \citeauthor{topic_modeling2020} have published surveys on common topic analysis techniques, 
including \ac{lda} and more advanced models which take into account factors such as time \cite{topic_modeling2015, topic_modeling2020}.

% similarity
To determine the similarity between two objects, one has to define a metric.
Prevalent metrics in the domain of comparing objects in a \ac{vsm} include (soft) cosine similarity 
as outlined by \citeauthor{soft_cosine2014} and \citeauthor{soft_cosine2017} \cite{soft_cosine2014, soft_cosine2017}
and the Euclidean norm \cite{euclidean_l2_norm2015}.
\citeauthor{euclidean_l2_norm2015} evaluate and compare different norms in the context of \ac{ir} from images \cite{euclidean_l2_norm2015}.


% clustering
The similarity between objects can be used to cluster them.
The employment of the clustering algorithm \ac{dbscan} on \ac{d2v} embeddings was proposed by \citeauthor{clusteringDocs2020} \cite{clusteringDocs2020}.
Other researchers, for instance, \citeauthor{OPTICS_kMeans_2016}, compare related clustering algorithms including \ac{dbscan} and \ac{optics}.
Initially, \ac{optics} was introduced by \citeauthor{OPTICS1999} in \citeyear{OPTICS1999} as a method to analyze the density-based structure of data from a database \cite{OPTICS1999}.
\citeauthor{OPTICS2013}, \citeauthor{OPTICS2014} and \citeauthor{OPTICS2016} propose \ac{optics} extensions, 
i.e.\ for spatially and temporally evolving data or a parallel version \cite{OPTICS2013, OPTICS2014, OPTICS2016}.


% dimensionality reduction
Since many algorithms suffer from the curse of dimensionality, i.e.\ their quality deteriorates with increasing dimensionality, 
\citeauthor{dim_reduction2021} have proposed a survey on different dimensionality reduction techniques \cite{dim_reduction2021}.
These techniques include \ac{pca}, \ac{lda} and \ac{svd}, which are well-known in the context of \ac{ir}.


% eigenfaces
A specific approach in the domain of face recognition is adopted in this thesis.
\eigenfaces{} projects face images into a lower-dimensional feature space which best encodes the variation among the faces \cite{eigenfaces1991}.
Since \citeyear{eigenfaces1991} this technique has been covered in a lot of papers 
\cite{eigenfaces1991, eigenfaces1997, eigenfaces2013, face-recognition2008, face-recognition2020, face-recognition2021}.
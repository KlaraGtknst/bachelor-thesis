% similar data corpus (fiscal data)


% similar problems
Working with huge amounts of textual data is a prevalent challenge in text-mining tasks.
\citeauthor{InformationRetrieval1999} propose a survey on \ac{ir} from documents.
They have explored methods to extract information from different formats, including text and images \cite{InformationRetrieval1999}.
\citeauthor{nlp-book2009} published a book covering the \ac{nlp} fundamentals using Python \cite{nlp-book2009}.
Common preprocessing techniques, including the ones from \autoref{sec:preprocessing} are outlined. 

\citeauthor{WordRep2013}'s work covers continuous vector representations of words from very large data sets \cite{WordRep2013}.


% models/ embeddings
\citeauthor{clusteringDocs2020} assess the \ac{d2v} model used in this work claiming it overcomes several of \ac{tfidf}'s shortcomings \cite{clusteringDocs2020}.
Similarly, \citeauthor{SentRep2014} argue that \ac{d2v} is superior to \ac{bow} models \cite{SentRep2014}.

\citeauthor{HfsentTrans2019} propose \ac{sbert} \cite{HfsentTrans2019}.

\citeauthor{inferSent2018} propose \infersent{} \cite{inferSent2018}.

\citeauthor{UniversalSentEnc2018} propose different \ac{use} architectures and compare their strengths \cite{UniversalSentEnc2018}.




% clustering
The usage of the clustering algorithm \ac{dbscan} on \ac{d2v} embeddings was proposed by \citeauthor{clusteringDocs2020} \cite{clusteringDocs2020}.

% dimensionality reduction
Since many algorithms suffer from the curse of dimensionality, \citeauthor{dim_reduction2021} proposed a survey on different dimensionality reduction techniques \cite{dim_reduction2021}.
These techniques include \ac{pca}, \ac{lda} and \ac{svd}, which are well-known in the context of \ac{ir}.
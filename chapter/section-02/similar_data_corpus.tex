% fundamentals
Since working with huge amounts of textual data is a prevalent challenge in text-mining tasks,
\citeauthor{InformationRetrieval1999} have published a survey on \ac{ir} from documents in which
they explore methods to extract information from different formats, including text and images \cite{InformationRetrieval1999}.
\citeauthor{nlp-book2009} published a book covering the \ac{nlp} fundamentals using Python \cite{nlp-book2009}.
Common preprocessing techniques, including the ones from \autoref{sec:preprocessing} are outlined. 


% models/ embeddings
In order to assess similarity among items of a corpus of textual data, documents have to be represented in terms of real-valued vectors.
\citeauthor{WordRep2013}'s work covers continuous vector representations of words from very large data sets \cite{WordRep2013}.
Specific models used in this work include 
the \ac{d2v} model, the \ac{sbert} model \cite{HfsentTrans2019}, the \infersent{} model \cite{inferSent2018} and the \ac{use} model.
Both \citeauthor{clusteringDocs2020} and \citeauthor{SentRep2014} assess the \ac{d2v} model, 
arguing that it overcomes several of \ac{tfidf}'s shortcomings \cite{clusteringDocs2020} and that it is superior to \ac{bow} models \cite{SentRep2014}.
\citeauthor{UniversalSentEnc2018} compare the strengths of different \ac{use} architectures \cite{UniversalSentEnc2018}.


% similarity
To determine the similarity between two objects, one has to define a metric.
Prevalent metrics in the domain of comparing objects in a \ac{vsm} include (soft) cosine similarity 
as outlined by \citeauthor{soft_cosine2014} and \citeauthor{soft_cosine2017} \cite{soft_cosine2014, soft_cosine2017}
and the Euclidean norm \cite{euclidean_l2_norm2015}.
\citeauthor{euclidean_l2_norm2015} evaluate and compare different norms in the context of \ac{ir} from images \cite{euclidean_l2_norm2015}.


% clustering
The usage of the clustering algorithm \ac{dbscan} on \ac{d2v} embeddings was proposed by \citeauthor{clusteringDocs2020} \cite{clusteringDocs2020}.
Other researchers, for instance, \citeauthor{OPTICS_kMeans_2016}, compare related clustering algorithms including \ac{dbscan} and \ac{optics}.
Initially, \ac{optics} was introduced by \citeauthor{OPTICS1999} in \citeyear{OPTICS1999} as a method to analyse the density-based structure of data from a database \cite{OPTICS1999}.
\citeauthor{OPTICS2013}, \citeauthor{OPTICS2014} and \citeauthor{OPTICS2016} propose \ac{optics} extensions, 
i.e. for spatially and temporally evolving data or a parallel version \cite{OPTICS2013, OPTICS2014, OPTICS2016}.


% dimensionality reduction
Since many algorithms suffer from the curse of dimensionality, \citeauthor{dim_reduction2021} have proposed a survey on different dimensionality reduction techniques \cite{dim_reduction2021}.
These techniques include \ac{pca}, \ac{lda} and \ac{svd}, which are well-known in the context of \ac{ir}.


% eigenfaces
A specific approach in the domain of face recognition is adopted in this thesis.
\eigenfaces{} projects face images into a lower-dimensional feature space which best encodes the variation among the faces \cite{eigenfaces1991}.
Since \citeyear{eigenfaces1991} this technique has been covered in a lot of papers 
\cite{eigenfaces1991, eigenfaces1997, eigenfaces2013, face-recognition2008, face-recognition2020, face-recognition2021}.

% similar data corpus (fiscal data)
% Seminar? Tax office?
Due to the fact that online financial interactions become more popular, the crime rate in this domain has increased as well.
Consequently, researchers have started developing anomaly detection techniques for financial data, such as credit card data 
\cite{credit_f_SOM2006, fd_ARIMA2021, cf_AE2020, AE_RF2021, dt_svm_2012, kaggle_ex2017}.
The work stated above and this thesis both concern the financial sector.
However, this thesis does not work on numerical data but on \acp{pdf}.
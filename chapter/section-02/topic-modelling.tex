\section{Topic Modelling}\label{sec:topic-modelling}

Since more and more text data emerges, methods to analyse and extract information from them become more important.
One of these methods is topic modelling.
It is used to discover groups of words with similar meanings in a text corpus \cite{topic_modeling2015}.
A topic is defined as a cluster of words that frequently occur together.
In other words, it is a probability distribution over a vocabulary.
Furthermore, a document is defined as a mixture of topics.
A document can thus be created by a distribution over topics and a distribution over words given the topics.
According to \citeauthor{topic_modeling2015}, topic models ignore the order of words in a document by relying on the bag-of-words assumption.
Common topic modelling algorithms are \ac{lsa}, \ac{plsa}, \ac{lda} and \ac{ctm} \cite{topic_modeling2015}.




\subsection{\ac{bertopic}}\label{subsec:bertopic}

\subsection{Top2Vec}\label{subsec:top2vec}

The approach \topTwovec{} was proprosed by \citeauthor{Top2Vec2020} \cite{Top2Vec2020}.
It addresses several problems of state-of-the-art topic modelling approaches, such as \ac{lda}.
Opposed to \ac{lda}, \topTwovec{} does not require the user to specify the number of topics $k$, 
i.e. it does not discretize the topic space into $k$ topics, 
and it does not require stop word removal or lemmatization.
Moreover, it considers the semantic meaning of words unlike \ac{lda}.
Unlike \ac{lda}, \topTwovec{} only associates one topic with a document.

% topics
The authors state that topics are continuous and can be described by different sets of words \cite{Top2Vec2020}.
Hence, topic modelling is defined as the task of finding sets of informative words that describe a document.
Documents in dense areas of the topic space are considered to be about the same topic.
The density-based clustering algorithm \ac{hdbscan} is used to find these dense areas.
The topic vector is denoted as the centroid or average of the document vectors belonging to the topic.
\citeauthor{Top2Vec2020} have tried alternative topic vector definitions, such as the geometric mean, finding the arithmetic mean to be the best option \cite{Top2Vec2020}.
The number of topics is derived from the number of dense areas.
It is possible to merge topics to hierarchically reduce the number of topics found to any number of topics smaller than the number initially found.

% word2vec and doc2vec
\topTwovec{} is based on \ac{w2v} and \ac{d2v}.
The learning task is to predict the document a word came from \cite{Top2Vec2020}.
Similar to \ac{d2v}, \topTwovec{} not only embeds words and documents in the same feature space but also topics.
\citeauthor{Top2Vec2020} view each point in the \ac{vsm} as a topic, described by its nearest words.

% dimensionality reduction
In order to find topics, clusters of documents, i.e. dense areas, need to be identified.
The clustering algorithm \ac{hdbscan} is used to find these dense areas.
Since \ac{hdbscan} has difficulties finding dense clusters in high-dimensional data, 
the dimensionality reduction method \ac{umap} is applied \cite{Top2Vec2020}.
The steps of the topic modelling procedure \topTwovec{} are depicted in \autoref{fig:top2vec}.

\begin{figure}[htp] % htp = hier (h), top (t), oder auf einer eigenen Seite (p).
    \centering
    \includesvg[width=0.7\textwidth]{images/topic_modelling/Top2Vec.svg}
    \caption{Procedure of topic modelling using \topTwovec{}.}
    \label{fig:top2vec}
\end{figure}


\subsection{\ac{lda}}\label{subsec:latent-dirichlet-allocation}

\ac{lda} is a generative model, which recreates the original document word distributions with minimal error given topics \cite{topic_modeling2015, Top2Vec2020}.
A discrete multinomial probability distribution over a vocabulary consisting of $W$ words is called a topic.
A document is a mixture of $K$ topics.
Hence, each document from the document corpus $D$ is represented by specific topic probabilities.

Due to the fact that high probability words are considered to be informative, \ac{lda} may regard words, 
which occur frequently as important even though they might be uninformative in reality \cite{Top2Vec2020}.

\cite{clusteringDocs2020}


\subsection{Word Clouds}\label{subsec:word-clouds}

\textcolor{red}{eher implementation}
The size of a word correlates to its frequency or importance in the text.
However, a word does not have to be meaningful to appear large.
A word cloud does not provide information about the meaning or context of words and thus, 
one has to be careful when interpreting the results.
The implementation of word clouds in this thesis is based on the Python library \textit{wordcloud} \cite{wordcloud-dev}.
This implementation removes certain English stop words from the text by default.
The input text is split into tokens using a regex.
By default, plurals are removed if their singular version is present and their frequency is added to their singular version.
By default, numbers are not included as phrases/ tokens.
% similar work (commercial search engines)

% BERTopic
Some researchers have already developed complete topic analysis libraries.
They merge a selection of the techniques stated above into a well-reasoned composite.
Best-known examples include BERTopic, a composition of \ac{sbert}, a dimension reduction using \ac{umap}, \ac{hdbscan} clustering and the application of \ac{tfidf} on the clusters \cite{bertopic2022}.
% The BERTopic approach outlined by \citeauthor{bertopic2022} consists of multiple components which are similar to the idea proposed in this thesis:
% Namely, using similarity measures on (different precomputed) embeddings and displaying the resulting clusters human interpretable.
% More concretely, the clustering algorithm used to group the documents based on visual similarity is related to \ac{hdbscan}.

% LDA
Other well-established topic analysis approaches consist of \ac{lda}.
\citeauthor{lda2008} propose a technique that first applies \ac{lda} to reduce the data's dimensionality and thereafter classifies the result with a \ac{svm} \cite{lda2008}.
Similarly, \citeauthor{LDA2016} use a \ac{knn} algorithm instead of a \ac{svm} on the textual subspace generated by \ac{lda} \cite{LDA2016}.
Another technique proposed is LDA2VEC, which is subject to \citeauthor{evolution_of_topic_modeling2022}'s work \cite{evolution_of_topic_modeling2022}.
\citeauthor{topic_modeling2021}'s paper introduces an open-source library for topic model visualization, exemplary showcased on a Wikipedia dataset \cite{topic_modeling2021}.

% Top2Vec
\citeauthor{Top2Vec2020} and \citeauthor{Topic2Vec2015} claim that the \ac{t2v} model not only overcomes \ac{lda}'s shortcomings \cite{Top2Vec2020, Topic2Vec2015}
but is also developed for topic analysis on a large collection of documents \cite{Top2Vec2020}.
The \ac{t2v} library serves as a baseline model for the application developed in this work.

% similar data corpus (fiscal data)
% Seminar? Tax office?
Due to the fact that online financial interactions become more popular, the crime rate in this domain has increased as well.
Consequently, researchers have started developing anomaly detection techniques for financial data, such as credit card data 
\cite{credit_f_SOM2006, fd_ARIMA2021, cf_AE2020, AE_RF2021, dt_svm_2012, kaggle_ex2017}.
The work stated above and this thesis both concern the financial sector.
%However, this thesis does not work on numerical data but on \acp{pdf}.
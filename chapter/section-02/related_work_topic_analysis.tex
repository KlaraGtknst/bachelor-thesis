The methods explored in this thesis ought to be bundled into a tool.
Some researchers have already developed complete topic analysis libraries whose functionalities can be compared to the tool developed in this thesis.
% BERTopic
They merge a selection of the techniques stated above into a well-reasoned composite.
BERTopic is a library that merges \acs*{sbert} embeddings with \acs*{umap} dimension reduction, 
\acs*{hdbscan} clustering and the application of \acs*{tfidf} on the clusters \cite{bertopic2022}.

% LDA
Other well-established topic analysis approaches consist of \acs*{lda}.
\citeauthor{lda2008} propose a technique that first applies \acs*{lda} to reduce the data's dimensionality 
and thereafter classifies the result with a \ac{svm} \cite{lda2008}.
Similarly, \citeauthor{LDA2016} use a \acs*{knn} algorithm instead of a \ac{svm} on the textual subspace generated by \acs*{lda} \cite{LDA2016}.
Another technique proposed is LDA2VEC, which is subject to \citeauthor{evolution_of_topic_modeling2022}'s work \cite{evolution_of_topic_modeling2022}.
\citeauthor{topic_modeling2021}'s paper introduces an open-source library for topic model visualization, 
exemplary showcased on a Wikipedia dataset \cite{topic_modeling2021}.

% Top2Vec
\citeauthor{Top2Vec2020} and \citeauthor{Topic2Vec2015} claim that the \acs*{t2v} model not only overcomes \acs*{lda}'s shortcomings \cite{Top2Vec2020, Topic2Vec2015}
but is also developed for topic analysis on a large collection of documents \cite{Top2Vec2020}.
The \acs*{t2v} library serves as a baseline model for the application developed in this work.

% time and dynamic
In contrast to the assumption of this thesis that the prevalent topics are static, in reality, topics may be dynamic or change over time.
Not only \citeauthor{topic_modeling2015}, but also \citeauthor{topic_modeling2020} have published surveys on topic analysis techniques, 
which take into account factors such as time \cite{topic_modeling2015, topic_modeling2020}.


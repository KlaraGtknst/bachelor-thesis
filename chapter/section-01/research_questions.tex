\section{Research Questions}\label{sec:research-questions}

In order to support individuals exploring large unstructured text data, this thesis aims to provide computational means aiming to facilitate the work with large unstructured text data.
In this work, different methods to derive semantic and visual information from unstructured text data are applied.
The techniques ought to be compared and evaluated.

% User Interface
In order to facilitate the exploration of the data, a \ac{ui} shall be provided.
However, this \ac{ui} is not supposed to be an operational application for end users from the tax office 
but serves the purpose of displaying the techniques worked on.
It should assist the natural human approach to exploration:
A human finds a document of interest, for instance, by random sampling, and thus, wants to find similar documents.

% goals
In the following, the goals of this work are defined.
\textcolor{red}{TODO: research QUESTIONS}

The following bullet points are considered secondary goals which have influenced the design process of this thesis.
\begin{description}
    \item[Evaluation of scalability.]
    The techniques applied should be evaluated in terms of their usability for large datasets, such as the Bahamas leak.
    \item[Similarity of grouped documents.]
    The document clusters should be derived from either semantically or visually similar documents.
    \item[Topic identification.]
    The topics identified should be meaningful to the task at hand and human interpretable.  
    \item[Little latency.]
    The database should be calculated offline to ensure little latency when executing queries.
    \item[Usability.]
    The methods should be bundled in an application, which is easy to use and does not require any programming skills.
\end{description}
\section{Research Questions}\label{sec:research-questions}

In order to support the exploration of large unstructured text data, 
this thesis aims to provide computational means to facilitate the work with large unstructured text data.
In this work, different methods to derive semantic and visual information from unstructured text data are applied.
These techniques ought to be compared and evaluated.
% research questions (allgemein damit nützlich und speziell für die Aufgabe -> beantwortbar, Erkenntnis allg Interesse & Operationalisierbar: Experiment)
In the following, the research questions addressed are defined:
\begin{questions}%[style=unboxed]
    \item Is it possible to use a visual representation to find similar documents in the corpus?\label{enum:rq1}
    
    \item How are the results of the system presented to experts?\label{enum:rq2}
    
    \item How can the performance of the system be evaluated?\label{enum:rq3}

    \item Do different embedding methods produce similar results?\label{enum:rq4}
    The task at hand defines a result as a set of similar documents to a query document.
    Hence, one has to compare response sets of different methods.
\end{questions}


\ref{enum:rq1} is answered by the implementation of the system.
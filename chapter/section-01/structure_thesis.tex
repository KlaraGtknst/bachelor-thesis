
\section{Structure of the Thesis}\label{sec:structure-of-the-thesis}
The thesis is structured as follows.

\begin{description}
    \item[Chapter \ref{ch:introduction}] \hfill \\
        Firstly, the problem of working with large unstructured text corpora is introduced.
        Secondly, the dataset used in this thesis is described.
        Moreover, the goal of this thesis, as well as
        the target audience of the problem investigated is stated.
        Afterwards, the motivation and research questions are presented.
        The chapter concludes with an outlook on the techniques used and an overview of the thesis.
    
    \item[Chapter \ref{ch:related-work}] \hfill \\
        This chapter covers related work where similar approaches are presented.
        Moreover, the chapter introduces the literature that serves as a basis for this thesis.

    \item[Chapter \ref{ch:methodology}] \hfill \\
        The theoretical foundations of the techniques applied in this thesis are outlined in \autoref{ch:methodology}.
        The techniques can be divided into preprocessing (cf. \autoref{sec:preprocessing}), 
        semantic embeddings (cf. \autoref{sec:embeddings}), 
        similarity measurements (cf. \autoref{sec:similarity-measurement}),
        topic analysis (cf. \autoref{sec:topic-modeling}),
        compression of data (cf. \autoref{sec:compression}),
        clustering algorithms (cf. \autoref{sec:clustering})
        and software frameworks (cf. \autoref{sec:frameworks}).

    \item[Chapter \ref{ch:implementation}] \hfill \\  
        This chapter describes the implementation of the methods.
        The implementation is based on the theoretical foundations presented in \autoref{ch:methodology}.
        On a more granular level, this chapter covers the offline preprocessing (cf. \autoref{sec:offline-processing}),
        the implementation of the \ac{ui} (cf. \autoref{sec:ui})
        and the trade-off between memory and query time in \autoref{sec:trade-off}.

    \item[Chapter \ref{ch:evaluation}] \hfill \\
        The evaluation of the methods is presented in this chapter.
        It gives a reason why certain parameter choices were made 
        with respect to established parameter estimation approaches.
        Moreover, it compares the different methods with regard to their query responses 
        and the bundle of methods constructed in the course of this thesis to an existing baseline topic analysis approach.

    \item[Chapter \ref{ch:conclusion}] \hfill \\
        This chapter concludes this thesis.
        The insights acquired by exploring different techniques with the goal of the exploration of large unstructured text data 
        are presented and the research questions are revised in \autoref{sec:discussion}.
        In \autoref{sec:contribution} the scientific contributions are highlighted.

    \item[Chapter \ref{ch:outlook}] \hfill \\
        The last chapter gives an outlook on future work.
        It also includes a discussion of the limitations of this thesis.

\end{description}
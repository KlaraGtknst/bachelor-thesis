\section{Motivation/ Objective}\label{sec:motivation}

Assumption: similarities between documents (in terms of appearance and content-wise)
On a broader scope, this thesis aims to provide computational means to facilitate the work with large unstructured text data for individuals.
In the following, certain goals are defined, which are to be achieved in this thesis.

Motivation/ problem: actively use machine learning techniques to analyse large text corpus and thus, reduce the amount of manual (human) work.
This includes analysis in terms of textual (content) and visual (appearance/ layout) information like a human would do.
The goal is to identify similarities between documents and group (cluster) them together - the topic of the cluster does not have to be labelled specifically.
This serves as a first step/ preprocessing, e.g., a human finds a document of interest (for instance from random sampling) and wants to find similar documents to it.


\begin{description}
    \item[Usability.]
    The methods should be bundled in an application, which is easy to use and does not require any programming skills.
    \item[Semantic similarity.]
    The documents grouped together should be semantically similar.
    \item[Topic identification.]
    The topics identified should be meaningful to the task at hand.  
    \item[Offline Calculation.]
    The database should be calculated offline so that the queries can be executed with little latency.
\end{description}
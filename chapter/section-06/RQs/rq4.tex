% RQ4: evaluation of performance
\paragraph{\ref{enum:rq4}} addresses the evaluation of the performance of the system.
% time measurements
In this work, the system is evaluated with respect to multiple parameters.
\autoref{subsec:evaluation-db} discusses 
the time to compute the different embeddings in \autoref{fig:time_init_db}.
In \autoref{sec:eval-embeddings}, some embedding models are evaluated 
with respect to their time consumption for different configurations (cf. \autoref{fig:times_emb}).

% venn diagram, heatmap, statistical properties
Another way to evaluate the performance of the system is to 
compare the results of different models via the intersection 
of their response sets (cf. \autoref{sec:evaluation-models}).
These intersections can be visualized with Venn diagrams (\autoref{fig:venn-comparison-models}) 
and heatmaps (\autoref{fig:heatmap-comparison-models}).
To ensure the results are not random, 
the statistical properties of the response sets are calculated 
and presented in \autoref{tab:mean_std_portion_shared_query_results}.
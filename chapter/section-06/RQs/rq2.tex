% RQ2: different embedding methods -> similar results?
\paragraph{\ref{enum:rq2}} aims to find out whether different embedding methods produce similar results.
% 2.1. Comparison of different semantic embedding methods
% semantic: TFIDF, Doc2Vec, USE, InferSent, SBERT
When comparing different semantic embedding methods in \autoref{sec:evaluation-models}, 
slight differences between the models become evident.
\ac{tfidf}, \ac{d2v} and \ac{sbert} are most dissimilar to each other.
The \ac{tfidf} approach performs rather poorly on unusual query documents such as handwritten ones.

% 2.2. Comparison of semantic and visual embedding methods.
The distinct response documents and their order are different for the same query document regarding different semantic embeddings.
Different semantic embeddings can yield response documents containing the same company name.
While the semantic responses' contents are considered similar to the query document and each other,
the visual responses are more dissimilar from the query document and each other.

% However, models such as \ac{tfidf} yield less meaningful results on certain query documents and 
% are prone to performance deterioration with increasing document corpus, since the embeddings' 
% dimensionality correlates directly with the vocabulary size.
% When the number of terms in the data corpus increases, but the vocabulary size is static, 
% the vocabulary likely lacks important words. 

To answer \ref{enum:rq2}, the content and visual appearance of the response documents 
of different embedding techniques is similar,
but the actual documents in the response sets differ.

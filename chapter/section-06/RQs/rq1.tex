% RQ1: visual
\paragraph{\ref{enum:rq1}} seeks to discover whether visual embedding methods are suitable 
for the task of finding similar documents in large unstructured text corpora.

In this work, firstly, the images are preprocessed using the \eigendocs{} approach as discussed in \autoref{subsec:impl-optics} and \autoref{sec:evaluation-OPTICS}. 
Then, the numerical data obtained from preprocessing the images is then reduced using \ac{pca}.
Determining the number of components to be used for \ac{pca} in a meaningful scientific way was problematic 
(cf. \autoref{subsec:eigenface} \& \autoref{sec:evaluation-eigendocs}):
The cumulative explained variance did not indicate a small number of components to use.
The \ac{rsme} calculation was found to be unsuitable for the dataset since 
multiple random selections of documents from the dataset did not indicate a clear "elbow" point.
% 1.2. OPTICS vs. argmax
The compressed images are clustered using the \ac{optics} algorithm and the \texttt{argmax} approach (cf. \autoref{sec:evaluation-models}).

% 1.3. overall quality of query responses
In general, the query responses from \autoref{sec:evaluation-models} which are based on visual information 
consist of visually similar documents.
Hence, in terms of qualitative evaluation, the visual representation is considered to be 
valuable means to find visually similar documents in large unstructured corpora.
Consequently, the answer to \ref{enum:rq1} is positive but 
acknowledges the lack of scientific justification for the proof of concept's configuration.
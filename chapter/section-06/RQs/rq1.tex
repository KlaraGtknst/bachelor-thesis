% RQ1: visual
\paragraph{\ref{enum:rq1}} seeks to discover whether visual embedding methods are suitable 
for the task of finding similar documents in large unstructured text corpora.
% 1.1. PCA components
In this work, the visual information of a document is encoded and stored in a database.
When comparing the literature approach of resizing the original images to 32x32 images with the 
\eigendocs{} approach for preprocessing the images (cf. \autoref{subsec:impl-optics}), 
the latter approach was found superior.
As discussed in \autoref{sec:evaluation-OPTICS}, 
this is due to the fact that the \eigendocs{} approach preserves more information than the literature approach.

The numerical data obtained from preprocessing the images is then reduced using \ac{pca}.
In order to determine the number of components to be used for the \ac{pca}, 
both the cumulative explained variance and the reconstruction error \ac{rsme} are considered 
(cf. \autoref{subsec:eigenface}).
Since the cumulative explained variance did not indicate a small number of components to be used, 
the \ac{rsme} was chosen to be more meaningful.
However, the \ac{rsme} calculation was found to be unsuitable for the dataset.
This hypothesis is motivated by multiple random selections of documents from the dataset that
did not indicate a clear "elbow" point.
The difficulty of determining the number of components to be used for the \ac{pca} in a meaningful scientific way 
do not satisfy the requirements of a scientific approach.

% 1.2. OPTICS vs. argmax
\autoref{sec:evaluation-models} compares the \ac{optics} approach 
with the \texttt{argmax} approach for two sample queries.
The \texttt{argmax} approach returns inferior results to \ac{optics} on the query document 
of type certificate in \autoref{fig:comp_vis_query_resp_certificates}.
Consequently, \ac{optics} is considered the superior clustering method.

% 1.3. overall quality of query responses
In general, the query responses which are based on visual information from \autoref{sec:evaluation-models} 
consist of visually similar documents.
In terms of qualitative visual evaluation, the visual representation is considered to be 
valuable means to find visually similar documents in large unstructured corpora.
Consequently, the answer to \ref{enum:rq1} is positive but 
acknowledges the lack of scientific justification for the proof of concept's configuration.
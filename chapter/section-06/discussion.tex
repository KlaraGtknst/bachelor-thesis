\subsection{Discussion}\label{subsec:discussion}

% preprocessing -> filter information
Before working with a model, the data has to be processed to a format that the model can work with.
The requirements imposed on the data format depend on the model.
Some models do not require any preprocessing, while others require a lot of preprocessing.

% tfidf preprocessing
Opposed to the other semantic embedding methods, \ac{tfidf} requires preprocessing.
In this work, the \ac{tfidf} configuration includes the definition of a custom preprocessor.
Consequently, preprocessing plays an important role in this embedding.
The preprocessor is constructed with respect to the well-established preprocessing steps for \ac{nlp} tasks \cite{nlp-book2009},
the \ac{tfidf} documentation about the default preprocessor and the task.
Since the data set can contain figures due to the financial background, the preprocessor encodes all numbers into discrete strings.
Hence, the information is not lost but strongly reduced, which is vital due to the constraints imposed 
by the database's dense vector dimension limitation.
The custom preprocessor is compared to the default preprocessor on multiple datasets.
Since it consistently produces smaller vocabulary sizes, it is chosen for the task.

% visual preprocessing
The visual embedding methods also require preprocessing.
Originally, the images were matrices of \texttt{RGB} values.
However, \ac{pca} and \ac{optics} require a vector as input.
Therefore, the images are converted to greyscale and flattened to a vector.
The images are resized to make them more comparable.
Since clustering algorithms such as \ac{optics} encounter difficulties on high-dimensional data, 
applying \ac{pca} to compress the data can be considered preprocessing.


% compression: AE, PCA/ Eigendocs
As mentioned before, the database has a dense vector dimension limitation of 2048.
This limitation is a problem for the visual embedding methods as well as for \infersent{} and 
\ac{tfidf}.
The visual embedding methods are compressed with \ac{pca}, 
whereas the encoder of a trained \ac{ae} is employed for textual embeddings.
The \ac{pca} is fitted on 1000 randomly selected images.
The number of components is selected beforehand in consideration of the explained variance and the reconstruction error.
The architecture of the \ac{ae} is chosen in consideration of the reconstruction error on 195 documents.
However, both the \ac{pca} and the \ac{ae} model's quality would improve if they were trained on a larger dataset.


% visual vs semantic
As already stated, this work examined both ways to encode semantic and visual information.
The evaluation is conducted with respect to the task of finding similar documents.
Since the data set is not labeled, the evaluation is subjective and experimental.
Qualitative evaluation is conducted by inspecting the responses of different models for several sample queries to the database 
storing the respective embeddings.
The inspection is conducted manually, 
using Venn diagrams, heatmaps and 
statistical properties of the distribution of the cardinality of the shared query response sets.

The semantic responses are similar and often managed to find either documents with the same content or documents with the same company name.
The visual responses are more dissimilar from each other and the query document.
% visual: OPTICS, argmax
More precisely, the \texttt{argmax} approach returns poor results to some extent.
Consequently, \ac{optics} is considered the superior visual embedding method for the database.
Moreover, the textual embedding methods produce more meaningful responses than the visual embedding methods.
For example, textual embedding methods manage to find documents with the same company name, 
while visual embedding methods naturally return visually similar documents which not necessarily originated by the same company.


% semantic: TFIDF, Doc2Vec, USE, InferSent, SBERT
When evaluating the semantic embedding methods, slight differences between the models become evident.
Differences and similarities between the models are examined by comparing the responses to the sample queries.
This evaluation includes the calculation of the average portion of shared documents between the responses of several models 
and the exemplary evaluation of actual queries.
\ac{tfidf}, \ac{d2v} and \ac{sbert} are most dissimilar from each other.
The \ac{tfidf} approach performs rather poorly on unusual query documents such as handwritten documents.


% similarity: cosine
The similarity metric used in this work is cosine similarity.
Since \databaseName{} provides the similarity metrics Euclidean distance, dot product and cosine similarity,
one of these metrics had to be chosen.
Usually, cosine similarity is recommended for \ac{nlp} tasks and thus, applied in this work.
Soft cosine similarity is not used, since it is not available in \databaseName{}.
However, the usage of soft cosine would likely improve the results.

% topic analysis: wordclouds
Since the techniques discussed in this section have to be merged into a single system, 
\wordcloud{} provides means to visualize the response documents containing the most similar documents for different models.
This approach is well suited to describe topics as groups of predominant words \cite{topic_modeling2019}.
The baseline topic analysis technique \ac{t2v} provides a \wordcloud{} implementation too.

% database: Elasticsearch
The database used in this work is \databaseName{}.
It was built to provide a fast and scalable search engine.
Moreover, it is a document-oriented database, which is well-suited for flexible data.
By default, several similarity measures and search strategies are provided.
Moreover, the elastic stack offers a wide range of tools supporting embedding models which are not used in this work.


% FE (Angular), BE (Flask)
The libraries used for the implementation of the tool are \angular{} and \flask{}.
The libraries were chosen after discussing options with this thesis' supervisor.
However, the implementation of this tool is not the focus of this work.

% top2vec
The baseline topic analysis technique \ac{t2v} is chosen due to its simplicity and functionalities.
The \ac{t2v} implementation is well documented and provides a \wordcloud{} implementation.
Moreover, it enables the user to find similar topics for query terms and returns inherent topics in a data corpus.
Hence, its functionality is similar to the functionality of the system developed in this work.
\chapter{Methodology}\label{ch:methodology}

\cite{InformationRetrieval1999}

Basic concepts, methods used, etc.

\section{Preprocessing}\label{sec:preprocessing}

\subsection{Tokenization/ Chunking}\label{subsec:tokenization}

\subsection{Lemmatization}\label{subsec:lemmatization}
Type of Stemmers. Porter, Snowball, Lancaster, etc.
% https://databasecamp.de/daten/stemming-lemmatizations for difference betwee stemmers and lemmatizers
Pre-trained/defined dense vector dictionaries (Word2Vec, \ac{glove}, FastText, etc.)

\subsection{Stop-Word-Removal}\label{subsec:stop-word-removal}

\subsection{Lower case}\label{subsec:lower-case}


\section{Similarity Measurement}\label{sec:similarity-measurement}

\cite{EmbDist2015}

\subsection{Cosine Similarity}\label{subsec:cosine-similarity}

\subsection{Soft Cosine Similarity}\label{subsec:soft-cosine-similarity}

\subsection{euclidian distance}\label{subsec:euclidian-distance}

%\subsection{Hamming distance}\label{subsec:hamming-distance}

%\subsection{\ac{wmd}}\label{subsec:word-mover-distance}

%\subsection{SpaCy}\label{subsec:spacy}


\section{Embeddings}\label{sec:embeddings}

\cite{WordRep2013}
\cite{SentRep2014}

\textcolor{red}{Skizze von Pipeline für jedes Embedding, welche zeigt, wie die Daten vorverarbeitet (stemming etc.) werden/ was das Model selber macht.}

%\subsection{\ac{cbow}}\label{subsec:bag-of-words}

\subsection{\ac{d2v}}\label{subsec:doc2vec}
\cite{SentRep2014}
two flavor of doc2vec: PV-DM and PV-DBOW (https://thinkinfi.com/simple-doc2vec-explained/)
\cite{SkipGram2013}

%\subsection{\ac{w2v}}\label{subsec:word2vec}

\subsection{\ac{tfidf}}\label{subsec:tfidf}
Test test test
% svg does not icons
\begin{figure}[h] % htp = hier (h), top (t), oder auf einer eigenen Seite (p).
    \centering
    \includesvg[width=1.0\textwidth]{images/TFIDF_embedding}
    \caption{TFIDF Preprocessing}
    \label{fig:tfidf_embedding}
\end{figure}

\begin{figure}[htp] % htp = hier (h), top (t), oder auf einer eigenen Seite (p).
    \centering
    \includesvg[width=1.0\textwidth]{images/TFIDF_preprocessing}
    \caption{TFIDF Preprocessing}
    \label{fig:preprocessing}
\end{figure}

\subsection{Universal sentence encoder}\label{subsec:univ-sent-encoder}
\ac{use}
\cite{UniversalSentEnc2018}

\subsection{InferSent}\label{subsec:inferSent}
\cite{inferSent2018}

\subsection{Hugging face's sentence Transformers}\label{subsec:hf-sent-ransformers}
\cite{HfsentTrans2019}

%\subsection{One Hot Encoding}\label{subsec:one-hot-encoding}

%\subsection{Skip Gram}\label{subsec:skip-gram}
%rather PV-DBOW, it is the document version of skip gram
% https://github.com/RaRe-Technologies/gensim/issues/1925
%\cite{SkipGram2013}

%\subsection{FastText}\label{subsec:fasttext}




\section{Topic Modelling}\label{sec:topic-modelling}

\subsection{\ac{bertopic}}\label{subsec:bertopic}

\subsection{\ac{lda}}\label{subsec:latent-dirichlet-allocation}

\subsection{Word Clouds}\label{subsec:word-clouds}
frequency of words in a document


\section{Appearance of documents}\label{sec:appearance}
documents saved as images in .png format, bad quality to minimize size of database
when querying db, top image results looked similar, which is how idea of this section arose

\subsection{Compression of data}\label{subsec:compression}
\subsubsection{AE}\label{subsec:autoencoder}

\subsubsection{eigenface}\label{subsec:eigenface}
like pca, but for images

\subsection{Clustering}\label{subsec:clustering}

Clustering is used in a variety of domains to group data into meaningful subsclasses, i.e. clusters \cite{OPTICS2013, OPTICS2014, OPTICS_kMeans_2016}.
According to \citeauthor{OPTICS2013}, common domains include anomaly/ outlier detection, noise filtering, document clustering and image segmentation. 
The goal is to find clusters, which have a low inter-class similarity and a high intra-class similarity \cite{OPTICS2013}.

There are multiple clustering techniques, which can be divided into four categories \cite{OPTICS2016}: 
\begin{enumerate}
    \item \textbf{Hierarchical clustering}:
    Algorithms, which create spherical or conex shaped clusters, possibly naturally occurring. 
    A terminal condition has to be defined beforehand.
    Examples include CLINK and SLINK \cite{OPTICS2014}.

    \item \textbf{Partitional based clustering}: 
    Algorithms, which partition the data into $k$ clusters, whereas $k$ is given apriori.
    Clusters are shaped in a spherical manner, are similar in size and not necessarily naturally occurring.
    KMeans is a popular example of a partitional based clustering algorithm.

    \item \textbf{Density based clustering}:
    Resulting clusters can be of arbitrary shape and size.
    The number of clusters does not have to be defined apriori.
    However, the algorithms are sensitive to input parameters, such as radius, minimum number of points and threshold.
    Popular examples are DBSCAN and OPTICS.
    
    \item \textbf{Grid based clustering}:
    Similar to density based clustering, but according to \citeauthor{OPTICS2016} better than density based clustering.
    Examples include flexible grid-based clustering \cite{OPTICS2014}.
    
\end{enumerate}

\subsubsection{KMeans}\label{subsec:kmeans}

\subsubsection{DBSCAN}\label{subsec:dbscan}


\subsubsection{HDBSCAN*}\label{subsec:hdbcan}


\subsubsection{OPTICS}\label{subsec:optics}

Opposed to DBSCAN, OPTICS is able to detect clusters of varying densities \cite{OPTICS2014}.
OPTICS produces an order of the elements according distance to the already added elements \cite{OPTICS2014}:
The first element added to the order list is arbitrary.
The order list is iteratively expanded by adding the element of the $\varepsilon$-neighbourhood of the order list items, which is has the smallest distance to any of the elements already in the order list.
When there are no more elements in the $\varepsilon$-neighbourhood to add, the process is repeated for the other clusters.



\subsubsection{Variational Bayesian estimation of a Gaussian mixture}\label{subsec:varbayes}

\subsubsection{Annoy}\label{subsec:annoy}



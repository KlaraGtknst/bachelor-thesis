\section{\eigendocs{}}\label{sec:evaluation-eigendocs}
% idea 
% Assuming the layout holds information about the document type, the first page of each document is used to extract this information.
% In the course of working with low-quality versions of the documents to minimize the memory necessary to store them, some documents looked similar.
% Therefore, the idea arose to use clustering algorithms to group the documents according to their appearance.

% number of components
In order to determine the optimal number of components used for \eigendocs{} the cumulative explained variance and the reconstruction error are plotted 
as displayed in \autoref{fig:det_n_comp} from \autoref{subsec:eigenface}.
The first plot indicates that 90\% of the variance is explained by 441 components.
Usually, that would have been the number of dimensions of the subspace onto which the documents would have been projected.
However, when working with clustering algorithms like \ac{optics}, the number of dimensions should be reduced even further to achieve valid clusters.
Therefore the reconstruction error with respect to different numbers of components is taken into consideration.

\begin{listing}[htp]
    \begin{minted}{python3}
        sqr_dif = (X_test - X_test_pca_inverse)**2
        reconstr_err.append(np.sqrt(np.mean(sqr_dif))/
            (np.sum(np.abs(1-X_test))/X_test.shape[0])) 
    \end{minted}
    \caption[Adaption of the \ac{rsme}]{
        Adaption of the \ac{rsme}: 
        Firstly, the squared differences between the original and the reconstructed images are calculated.
        Since the values are normalized, a 1 corresponds to a white pixel.
        Then, the absolute values of all non-white pixels of the test set are summed up.
        The average number of non-white pixels is calculated by dividing the sum by the number of images in the test set.
        This approach considers pixels of value $p \in [0,1]$ as $(p \cdot 100)$\% white and thus, they are incorporated in the sum.
    }
    \label{lst:impl-weighted-rsme}
\end{listing}

Usually, a \ac{rsme} is minimized to determine the optimal parameter configurations.
In this case, the reconstruction error shall be interpreted.
To facilitate the interpretability of the reconstruction error, its calculation is adapted to incorporate the content of the images.
At first sight, the majority of image pixels are white, i.e.\ do not convey any information.
Therefore, the reconstruction error is divided by the average number of non-white pixels. 
Hence, the reconstruction error of an image is weighted by the amount of information it conveys.
The calculation is given in \lst{lst:impl-weighted-rsme}.
The result is displayed in \autoref{fig:det_n_comp}.
Since the reconstruction error increases less rapidly after 10 to 20 components, the number of components is set to 13,
which has been an "elbow" point in a similar trial using a not randomly selected dataset of 195 images.


% results
Some impressions of the \eigendocs{} algorithm are displayed in \autoref{fig:preprocessed_docs_eigendocs}.
Assuming that the selection of documents is representative, 
the preprocessing of the documents using \eigendocs{} should have encoded information about the dimensionality of the images.
However, this assumption is not valid since bigger images exist.
Therefore, the idea of incorporating information about the image's dimensions is not entirely implemented.
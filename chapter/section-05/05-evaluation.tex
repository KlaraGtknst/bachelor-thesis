\chapter{Experimental evaluation}\label{ch:evaluation}

Since the dataset has no ground truth the procedure used to pick the parameter values is not comparable to ground truth-based approaches.
Hence, the evaluation is informal and the methods applied have arisen from regular consultation with experts from the tax office.

 % Database
 \section{Database}\label{sec:eval-db}
There is a variety of parameter values to choose from when working with databases and embeddings.
These parameters include similarity metrics and the choice of queries.

 \section{Similarity measurements}\label{sec:evaluation-sim-measurements}

According to \citeauthor{HfsentTrans2019}, the similarity measurements discussed above obtained roughly the same results in their experiments \cite{HfsentTrans2019}.   


 \subsection*{\databaseName{}}\label{subsec:evaluation-db}
% SQL vs NoSQL
According to \citeauthor{flask_book2018}, \ac{sql} databases are a good choice for efficiently storing structured data.
This is because their paradigm \acs{acid}, i.e.\ \acl{acid}, provides high reliability.
\ac{nosql} databases, on the other hand, are more flexible and can be used to store unstructured data \cite{flask_book2018}.
They do not require a predefined schema and can therefore accept documents of arbitrary structure \cite{flask2018}.
Usually, \ac{nosql} databases do not offer services such as \texttt{JOIN}s.
%According to \citeauthor{flask2018}, \ac{nosql} databases make a tradeoff between storage and speed, as well as a tradeoff between consistency and availability.
\ac{nosql} databases are said to outperform out-of-the-box \ac{sql} databases.
Since the dataset consists of unstructured documents and the task at hand does not require performing any \texttt{JOIN}s, 
a \ac{nosql} database is favourable.
\databaseName{} is chosen since it is well known to provide near real-time search and to operate on big data.
Subsequently, it is a good fit for the underlying dataset.

% limited dimensionality elastic search
Since \databaseName{} stores vectors of at most 2048 dimensions,
the \ac{tfidf} and \infersent{} embeddings are problematic.
Besides imposing limits to the dimensionality of the embeddings, \databaseName{} offers a variety of convenient functionalities,
such as the built-in \ac{knn} search.
Therefore, in this work, \databaseName{} is used regardless of the dimensionality constraints imposed by the database.
Hence, the techniques are adjusted to the database and not vice versa.

% separation of initialisation and insertion
The time necessary to fill the \databaseName{} database has been evaluated and improved throughout this work.
The current time measurements are shown in \autoref{fig:time_init_db}.
The times correspond to calculation of 2048 embeddings.
It is possible to measure this compuation time individually since the task of filling the database is modulized. 
Modulizing is beneficial since it is possible to update the embeddings without having to recreate the database. 
Moreover, it facilitates debugging and comparing the models used to create the embeddings.

\begin{figure}[!htp] % htp = hier (h), top (t), oder auf einer eigenen Seite (p).
    \centering
    \includegraphics[width=1\textwidth]{images/Elasticsearch/Time_necessary_to_compute_2048_embeddings_log.pdf}
    \caption[Times for creating the database]{Time per module of creating the Bahamas database using a random selection of 2048 documents.
    The x-axis is logarithmic.
    The reference time is measured using \texttt{cProfiler} on a \localMaschineStats{}.
    %the \texttt{timer} from \texttt{timeit.default\_timer} on a \localMaschineStats{}.
    }
    \label{fig:time_init_db}
\end{figure}

 % Eigendocs
 \section{\eigendocs{}}\label{sec:evaluation-eigendocs}
% idea 
Assuming the layout holds information about the document type, the first page of each document is used to extract this information.
In the course of working with low-quality versions of the documents to minimize the memory necessary to store them, some documents looked similar.
Therefore, the idea arose to use clustering algorithms to group the documents according to their appearance.

% number of components
In order to determine the optimal number of components used for \eigendocs{} the cumulative explained variance and the reconstruction error were plotted 
as displayed in \autoref{fig:det_n_comp} from \autoref{subsec:eigenface}.
The first plot indicated that 90\% of the variance is explained by 95 components.
Usually, that would have been the number of dimensions of the subspace onto which the documents would have been projected.
However, when working with cluster algorithms like \ac{optics} the number of dimensions should be reduced even further to achieve valid clusters.
Therefore the reconstruction error with respect to different numbers of components was taken into consideration.
The "elbow" points are visible at \textcolor{red}{10 and 13}.
Since visual inspection accounted for the fact that the decline of the reconstruction error after 13 was steeper than after 10, the number of components chosen is 13.

\begin{figure}[htp] % htp = hier (h), top (t), oder auf einer eigenen Seite (p).
    \centering
    \includegraphics[width=0.7\textwidth]{images/Eigendocs/transformation/eigendocs_13dims.pdf}
    \caption[The first 10 documents of the dataset]{The first 10 preprocessed documents of the dataset.
    The original images are displayed in the first row.
    The second row shows the reconstruction from their compressed version in the fourth row.
    The third row shows the reconstruction error, i.e. the difference between the reconstructed and the original image.
    The last row presents the greyscale values of the compressed 13-dimensional image as a line.
    }
    \label{fig:preprocessed_docs_eigendocs}
\end{figure}

% results
The results of the \eigendocs{} algorithm are displayed in \autoref{fig:preprocessed_docs_eigendocs}.
Assuming that the selection of documents is representative, 
the preprocessing of the documents using \eigendocs{} should have encoded information about the dimensionality of the images.
However, this assumption is not valid since bigger document images exist.
Therefore, the idea of incorporating information about the image's dimensions is not entirely implemented.
 
 % Embeddings
\section{Embeddings}\label{sec:eval-embeddings}
As discussed in \autoref{subsec:impl-embeddings}, there is a range of possible parameter values to choose from when implementing embedding models.
The section below states which findings have led to the parameter values applied in this work.

\section{Evaluation of \acs{tfidf}}\label{sec:evaluation-tfidf}

The main obstacle to overcome was the high dimensionality of the \ac{tfidf} embeddings.
Hence, the goal of the parameter selection was to find a way to reduce the dimensionality of the vocabulary to the maximum vector dimensionality of \databaseName{}.
However, the quality of the embeddings should not decline too much.

% parameter selection
The choice of the preprocessor was investigated with regard to the goal of minimizing the vocabulary size.
Both the default and custom preprocessor were tested on a data corpus of 195 documents with regard to the vocabulary (size) and the result of preprocessing.
A visualization obtained from the comparison is given in \autoref{fig:differences-vocabularies}.
While the default preprocessor had a vocabulary size of 1641, the custom preprocessor had a size of 1521.
The custom preprocessor was chosen because it had a smaller vocabulary size.
The differences between both vocabularies were compared and visualized.
The custom vocabulary has some bigrams, which are not present in the vocabulary produced by the default preprocessor.
Initially, the idea was to have a vocabulary that consists only of unigrams.

\begin{figure}%
    \centering
    \subfloat[\centering The terms only present in the vocabulary produced by the default preprocessor.]{{\includegraphics[width=6.5cm]{images/embeddings/tfidf/Word_cloud_in_vocabular_which_is_in_default_but_not_custom.pdf} }}%
    \qquad
    \subfloat[\centering The terms only present in the vocabulary obtained from the custom preprocessor.]{{\includegraphics[width=6.5cm]{images/embeddings/tfidf/Word_cloud_in_vocabular_which_is_in_custom_but_not_default.pdf} }}%
    \caption{The WordClouds visualize which words are not shared by both vocabularies.}%
    \label{fig:differences-vocabularies}%
\end{figure}

% two fields in db
Initially, there should have been two different \ac{tfidf} models.
The first one should have been used to obtain documents which are similar to the query document.
Therefore, terms which occur only once in the corpus should have been removed from the vocabulary.
The second approach should have been used to obtain specific documents from the corpus.
Hence, the vocabulary should consist of very document-specific terms and thus, \texttt{max\_df} would have been relatively low, to omit terms that occur in many documents.
However, the restrictions imposed by the database implementation made it impossible to explore many parameter ranges.
Therefore, only one \ac{tfidf} model was used in the end, whose parameters \texttt{min\_df} and \texttt{max\_df} were set to values which kept the vocabulary and thus,
the dimensionality of the embeddings reasonably small.

\subsection*{\ac{d2v}}\label{subsec:evaluation-doc2vec}

Since no labeled data is available, the evaluation of the \ac{d2v} embeddings is limited.
Therefore, the \ac{d2v} embeddings are evaluated by comparing them to other embeddings.
The \ac{d2v} model is not tuned in terms of hyperparameter selection,
but the default settings are used since there is no way to evaluate the resulting embeddings.


\section{\infersent{}}\label{sec:evaluation-inferSent}

% pool type
The \texttt{max} pooling type is used for the \infersent{} model, since \citeauthor{inferSent2018} 
found by conducting experiments using different pooling techniques that it was the best option.

% version/ embeddings dictionary
Initially, in this work, the \ac{glove} word embeddings were used for the \infersent{} model.
However, since the file of precomputed \acs{glove} word embeddings has a size of 5.65 \ac{gb} and thus,
slows down the model, ultimately another word embedding was used.
The time necessary to compute and insert 195 documents for specific embeddings is displayed in \autoref{fig:times_emb}.
The custom word embedding used in this work is a \ac{w2v} model trained on a selection of 195 documents from the Bahamas dataset.

% glove
\citeauthor{glove2014} state that \acs{glove} outperforms \ac{w2v} on the same corpus, 
vocabulary and window size in terms of quality and speed \cite{glove2014}.
Hence, the quality of the results obtained in this work may have suffered from using a custom \ac{w2v} instead of \acs{glove}.
However, since the computation time of the project is a crucial factor, the custom \ac{w2v} was used.

\begin{figure}%
    \centering
    \includegraphics[width=0.6\textwidth]{images/embeddings/infersent/InferSent_time_per_embedding.pdf}
    \caption{Time necessary to calculate and insert \infersent{} embeddings for different precomputed word embeddings on a \localMaschineStats{}.
    }
    \label{fig:times_emb}%
\end{figure}


\subsection*{Evaluation of \ac{use}}\label{subsec:evaluation-use}

Since there are no parameters to customize the evaluation of the \ac{use} embeddings is limited.
Therefore, the \ac{use} embeddings are evaluated by comparing them to other embeddings.

\subsection*{\ac{ae}}\label{subsec:evaluation-ae}

% architecture
In order to determine, which architecture for the hidden or so-called latent space of the \ac{ae} is the best option, 
different architectures were tested and compared in terms of \ac{rsme} and cosine similarity.
The \ac{rsme} is calculated as given in \lst{lst:impl-rsme}.
The cosine similarity is calculated as given in \lst{lst:impl-cos_sim}.
Due to the fact that cosine similarity values are bound by 0 and 1, they are easier to rank than metrics that can yield any real number.
However, cosine similarity is usually applied to calculate the angle between two vectors and thus, one has to be cautious when interpreting the result obtained.
For instance, the vectors $\left( 0, 1 \right)^T$ and $\left( 0, 2 \right)^T$ have a cosine similarity of 1, even though they are not the same vectors.
Since an \ac{ae} is supposed to reconstruct the input rather than return a dependent or related vector, this metric should be combined with a tarditional metric.
The dataset used for the evaluation is a selection of 195 documents from the Bahamas dataset.

% RSME
\begin{listing}[htp]
    \begin{minted}{python3}
        rsme = np.linalg.norm(inverse_embedding - embeddings) 
                / np.sqrt(embeddings.shape[0])
    \end{minted}
    \caption[Computation of the \ac{rsme}]{
        Computation of the \ac{rsme} between the original and the reconstructed embedding.
    }
    \label{lst:impl-rsme}
\end{listing}

% cosine similarity
\begin{listing}[htp]
    \begin{minted}{python3}
        cos_sim = statistics.mean([np.dot(inverse_emb, embedding)
                /(np.linalg.norm(inverse_emb)*np.linalg.norm(embedding)) 
                for inverse_emb, embedding in zip(inverse_embedding, embeddings)])
    \end{minted}
    \caption[Computation of the cosine similarity]{
        Computation of the cosine similarity between the original and the reconstructed embedding.
    }
    \label{lst:impl-cos_sim}
\end{listing}

The scores of different architectures are shown in \fig{fig:eval-ae-architecture}.
While most of the architectures produced similar results, one architecture stood out.
Combining 2500-, 3000- and 3500-dimensional layers in the hidden space produced the worst \ac{rsme} results.
The best results were achieved by adding a 3500-dimensional layer in the hidden space.
However, the results of the best architecture do not differ greatly from the others.

\begin{figure}[!htb] % htp = hier (h), top (t), oder auf einer eigenen Seite (p).
    \centering
    \includegraphics[width=1\textwidth]{images/embeddings/autoencoder/ae_score_plot.pdf}
    \caption[Different \ac{ae} architectures and their reconstruction error]{The effect of different \ac{ae} architectures on the reconstruction error.
    The error is measured in terms of \ac{rsme} (blue bars) and cosine similarity (yellow bars) between the original and the reconstructed image.
    The smallest \ac{rsme} and the biggest cosine similarity belong to the architecture best suited to this task and are coloured green.
    }
    \label{fig:eval-ae-architecture}
\end{figure}

% Clustering
\section{Clustering using \acs{optics}}\label{sec:evaluation-OPTICS}
% 3d plots
The algorithm \ac{optics} was applied to data, which was preprocessed according to \autoref{pt:32} and \autoref{pt:eigendocs}.
The clusters from \autoref{fig:optics_cluster} were extracted from the respective reachability plots in \autoref{fig:reachability_plots}.
The three-dimensional plots visualize the first three dimensions of the data and thus, the weights of the first three principal components assigned by the \eigendocs{} algorithm.
By visual inspection and comparison of both plots, it can be seen that the projection by the combination of resizing and \ac{pca} of \autoref{pt:32} scatters the objects further along the $x_2$ axis.
Hence, the distance between the objects is larger and more clusters are identified.
One could argue that the narrow distribution of the objects in the \eigendocs{} plot is due to the fact, 
that the input data encodes not only the visual appearance in terms of page layout but also the size of the document.
Possibly, this could explain why the objects are less scattered along this dimension.

% OPTIC cluster results
\begin{figure}%
    \centering
    \subfloat[\centering Preprocessing according to \autoref{pt:32}.]{{\includegraphics[width=5cm]{images/OPTICS/32x32/OPTICS_cluster_32x32.pdf} }}%
    \qquad
    \subfloat[\centering Preprocessing according to \autoref{pt:eigendocs}.]{{\includegraphics[width=5cm]{images/OPTICS/eigendocs/OPTICS_cluster_eigendocs.pdf} }}%
    \caption[\ac{optics} clusters]{The clusters were extracted from the respective reachability plots in \autoref{fig:reachability_plots} by \ac{optics}.
    The blue points are noise points, whereas any other colour denotes a cluster.}%
    \label{fig:optics_cluster}%
\end{figure}



% cluster content
To analyse the results of the clustering, the content of the clusters was examined.
Since the documents are not labelled, the content of the clusters was analysed by visual inspection.
The content of the clusters is displayed in \autoref{fig:clusters_32x32} and \autoref{fig:clusters_eigendocs_with_noise}.
The yellow images belong to the group identified as noise.
The images preprocessed according to \autoref{pt:32} were partitioned into multiple small and one big cluster.
The \eigendocs{} images' clusters have similar sizes. 
The row of noise images is thus, way longer than the other rows in \autoref{fig:clusters_eigendocs_with_noise}.
Most of the certificates are classified as noise for both approaches.


% 32x32
\begin{figure}[htp] % htp = hier (h), top (t), oder auf einer eigenen Seite (p).
    \centering
    \includegraphics[width=1.05\textwidth]{images/OPTICS/32x32/cluster_content_32x32.pdf}
    \caption[Detailed \ac{optics} clusters using 32x32 greyscale pixels]{The yellow images belong to the group denoted noise.
    Most certificates are classified as noise.
    There is one big cluster and multiple small clusters.
    The images were preprocessed as discussed in \autoref{pt:32} to 32x32 greyscale pixels.
    }
    \label{fig:clusters_32x32}
\end{figure}

% eigendocs with noise
\begin{figure}[htp] % htp = hier (h), top (t), oder auf einer eigenen Seite (p).
    \centering
    \includegraphics[width=1.05\textwidth]{images/OPTICS/eigendocs/cluster_content_incl_noise_Eigendocs.pdf}
    \caption[Detailed \ac{optics} clusters using \eigendocs{}]{Most certificates are classified as noise. The rest of the clusters have similar sizes.
    The images were preprocessed as discussed in \autoref{pt:eigendocs} to 13-dimensional greyscale pixels.
    }
    \label{fig:clusters_eigendocs_with_noise}
\end{figure}


The preprocessing approach used to create the \ac{optics} input for the \databaseName{} database index is \eigendocs{} since it also encodes information about the document size. 

% code
According to \citeauthor{OPTICS2014}, \ac{optics} was developed to improve \ac{dbscan} flaws.
Therefore, \ac{dbscan} is chosen for the cluster method in \lst{lst:optics_model}, since the literature consulted works with \ac{dbscan} as a basis.
In order to reduce calculation complexity the maximum $\varepsilon$ is 10.
The distance between two points to still be considered neighbours is defined after visual inspection of the reachability plot.
Considering the intrinsic structure of the data it is set to $0.5$ to return meaningful clusters.





\section{Comparison of models}\label{sec:evaluation-models}

% parameters
Similar to \citeauthor{glove2014}'s work, in this work, for many models used, any unspecified parameters are set to their default values, 
assuming that they are close to optimal
acknowledging that this simplification should be revised in a more thorough analysis.

% comparing models (qualitative)
difference query responses for different models?
any images which produce unusual results?

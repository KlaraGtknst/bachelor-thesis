\section{Comparison with baseline topic analysis approach}\label{sec:evaluation-top-model-app}

The baseline topic analysis \ac{t2v} offers a variety of built-in functionalities to the user.
It is possible to retrieve human interpretable inherent topics of a set of documents, 
as well as the topics most similar to certain search terms 
and \wordcloud{}s of these results.
Hence, this library meets the needs articulated by this work.

% > 1 embedding model
Opposed to \ac{t2v}, this thesis proposes a composite of different approaches to encoding visual and semantic information 
and query for them using a database and visualization by the means of \wordcloud{}s.
To be more specific, this thesis not only relies on one semantic embedding model but offers several techniques and an approach to incorporate visual information.

% no topics for search terms
However, it is not possible to query for topics of the corpus which best describe a search term.
Alternatively, one can perform a fuzzy text search on the documents.
The user can inspect the PDF of a document upon clicking on its name in the list of documents.
The detail view enables the investigation of similar documents in terms of different embedding approaches.

% topic definition
Due to \ac{t2v}'s architecture, documents and words are mapped into the same \ac{vsm}.
Hence, the topic vector definition and representation by its closest words are more meaningful than the approach of the thesis. 
In this thesis, a topic is represented by frequent words in the set of documents that are not necessarily meaningful. 

% term frequency
The tool implemented in this thesis can display the term frequency of the document chosen in the detail component.
The \ac{t2v} library does not offer a comparable service.
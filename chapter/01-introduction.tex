\chapter{Introduction}\label{ch:introduction}

According to \cite{data-corpus-bahamas-leaks}, the Bahamas leak is roughly 38 GB collection of documents, which were leaked from in 2016.
The data is used by (German) tax offices to identify tax evasion.
However, it has proven to be challenging to identify the relevant documents and connections between documents due to the amount of documents in the leak.

Therefore, the goal of this thesis is to suggest approaches to support the investigators of the tax offices.
Text exploration methods include topic modelling.

The topics to be identified can be groups of words which appear more often than the average or groups of similar documents.
Hence, a topic is not always the defined topic in terms of content, but sometimes a statistical phenomenon.
Since different methods define different topics, as they work and define the meaning of 'topic' differently, 
their results are compared and evaluated on the dataset.

Besides literature research, application and evaluation of the methods identified, 
certain preprocessing methods have proven to be eminent to successful work with unstructured text data.
These methods include chunking/ tokenization (separating texts into equally sized segments), lemmatization (e.g., faster to fast), 
conversion to small letters and stop-word-lists.

\section{Motivation/ Objective}\label{sec:motivation}
Zielsetzung


\section{Related work}\label{sec:related-work}


\section{Research Questions}\label{sec:research-questions}

list of research questions
\subsection[\acs{rq}1]{\ac{rq}1: Question 1?}\label{subsec:rq1}
explanation of \ac{rq}1


\section{Structure of the Thesis}\label{sec:structure-of-the-thesis}
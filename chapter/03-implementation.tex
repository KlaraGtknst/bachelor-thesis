\newcommand{\databaseName}{Elasticsearch}
\chapter{Implementation}\label{ch:implementation}

\section{Database Elasticsearch}\label{subsec:db}
% introduction, users
\databaseName{} is a widely used non-relational database, which was designed to store and perform full-text search on large corpus of unstructured data \cite{Elasticsearch2017}.
This open-source distributed document-driven database system is build in Java and is based on the Apache Lucene (Java) library for high-speed full-text search \cite{Elasticsearch2017,Elasticsearch2019}.
According to \cite{Elasticsearch2019}, \databaseName{} provides Wikipedia's full-text search and suggestions as well as Github's code search and Stack Overflow's geolocation queries and related questions.
Needless to say, \databaseName{} is prone to handle Big Data and enables near real-time search by index refreshing periods of one second.

% structure
\databaseName{}'s entries, i.e. documents are stored in logical units, so called indices.
% index
As stated in \citeauthor{Elasticsearch2019} and \citeauthor{Elasticsearch2017}'s work, the indices are structured similar to Apache Lucene's inverted index format.
An index can be spread into multiple nodes.
A node is single running instance of \databaseName{} \cite{Elasticsearch2019}.
An index is divided into one or more shards, which can be stored on different servers and enable parallelization \cite{Elasticsearch2019}.

% document
\databaseName{} indices' entries are documents, which are saved in a \ac{json} fromat \cite{Elasticsearch2017}.
A document's fields and field types are defined by the user when initializing the database index.
By default, every field of a document is indexed and searchable \cite{Elasticsearch2019}.

% Replicas
Replicas are copies of shards, which create redundancy and thus, ensure availability \cite{Elasticsearch2019}.


% content
The database is filled once with data from a large unstructured corpus of \acs{pdf}.
After the initialization of the database, it is used for queries. 
Therefore, the procedure is carries out in a completely offline fashion.

The index \textit{Bahamas} stores different embeddings of the text layer information and metadata of the documents.


% query (endpoints)
% get

% search id

% full-text search

% knn-search


columns are embeddings, rows are documents

\begin{figure}[htp] % htp = hier (h), top (t), oder auf einer eigenen Seite (p).
    \centering
    \includesvg[width=0.5\textwidth]{images/PDFs_to_database}
    \caption{Das Logo der Uni Kassel}
    \label{fig:pdf2db}
\end{figure}

\begin{figure}[htp] % htp = hier (h), top (t), oder auf einer eigenen Seite (p).
    \centering
    \includesvg[width=1.0\textwidth]{images/TFIDF_preprocessing}
    \caption{TFIDF Preprocessing}
    \label{fig:preprocessing}
\end{figure}

\section{User Interface}\label{sec:ui}
\chapter*{Zusammenfassung}
\markboth{Zusammenfassung}{Zusammenfassung}

Das Auffinden relevanter Dokumente und von Zusammenhängen zwischen Dokumenten wird durch die enorme Menge an verfügbaren Dokumenten erheblich erschwert.
Institutionen, wie z.B. deutsche Finanzämter, haben Zugang zu Datenleaks, wie etwa den \textit{Panama Papern} oder dem \textit{Bahamas-Leak}, 
die große Mengen an Dokumenten und wertvollen Informationen enthalten, die es zu extrahieren gilt.
Diese Institute, Unternehmen und Einzelpersonen verfügen jedoch nicht über ausreichende Ressourcen, um einzelne Dokumente 
zu durchsuchen, ein bestimmtes Dokument zu finden oder inhärente Themen zu identifizieren.
Daher können computergestützte Verfahren wie \textit{Text Mining} oder \textit{Topic analysis} sie dabei unterstützen.
In dieser Arbeit wird ein Ansatz vorgestellt, der in einem großen unstrukturierten Textkorpus relevante Dokumente mit gemeinsamen Themen findet.
Dieser Ansatz bündelt verschiedene Methoden, wie z.B. textuelle Embeddings, Transformation von Bildern und Clustering-Techniken.
Als Ergebnis der in dieser Arbeit untersuchten Methoden wird eine Weboberfläche bereitgestellt, die Abfragen nach ähnlichen Dokumenten 
zum Vergleich der verschiedenen Methoden ermöglicht.